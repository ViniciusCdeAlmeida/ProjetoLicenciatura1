\documentclass[10pt]{beamer}
\usepackage[utf8]{inputenc} % codificacao de caracteres
\usepackage[T1]{fontenc}    % codificacao de fontes
\usepackage[brazil]{babel}  % idioma
\usetheme{Antibes}         % tema
\usecolortheme{beaver}      % cores
\usepackage{listings}
\usepackage{color}
\usefonttheme[onlymath]{serif} % fonte modo matematic


\definecolor{gray}{rgb}{0.4,0.4,0.4}
\definecolor{darkblue}{rgb}{0.0,0.0,0.6}
\definecolor{cyan}{rgb}{0.0,0.6,0.6}

\lstset{
	basicstyle=\ttfamily,
	columns=fullflexible,
	showstringspaces=false,
	commentstyle=\color{gray}\upshape
}

\lstdefinelanguage{XML}
{
	morestring=[b]",
	morestring=[s]{>}{<},
	morecomment=[s]{<?}{?>},
	stringstyle=\color{black},
	identifierstyle=\color{darkblue},
	keywordstyle=\color{cyan},
	morekeywords={xmlns,version,type}% list your attributes here
}


%Titulo
\title[\sc{Static Analisys}]{Static analysis to detect evolution language in java projects}
\institute{
	\includegraphics[scale=0.7]{as_comp_cor.eps}\\
	\large
	\textbf{
	Universidade de Brasília\\
	Instituto de Ciências Exatas\\
	Departamento de Ciência da Computação}
	}
\date{\today}

\begin{document}
	
	\begin{frame}
		\titlepage
	\end{frame}
	
	\begin{frame}\frametitle{Roteiro}
		\begin{itemize}
			\item Spring
			\item Configurações
			\item Annotationss
			\item Beans
			\item XML
			\item Código
		\end{itemize}
	\end{frame}
	
	
	\begin{frame}\frametitle{Spring}
		\begin{itemize}
			\item Framework criado por Rod Johnson, em meados de 2002, com intuito de simplificar aplicações que antes só era possível com EJB's.\\
			\item Atualmente tem diversos módulos atenção especial para CDI.
		\end{itemize}
	\end{frame}
	


	\begin{frame}\frametitle{Configurações}
		\begin{itemize}
			\item spring-aop-4.1.6.RELEASE.jar
			\item spring-beans-4.1.6.RELEASE.jar
			\item spring-context-4.1.6.RELEASE.jar
			\item spring-core-4.1.6.RELEASE.jar
			\item spring-expression-4.1.6.RELEASE.jar
		\end{itemize}
	\end{frame}
	
	
	
	\begin{frame}\frametitle{Annotations}
		\textbf{@Autowired} para injetar dependêcia. A classe X tem um dependência da classe Y.\\
		E com interface com mais de uma classe implementando?\\
        \textbf{@Autowired}\\
		\textbf{@Qualifier("caixa1")}\\
		public class....
	\end{frame}
	

	\begin{frame}[fragile]\frametitle{Beans}
		\centering
		São as classes que serão injetadas.\\
		Em nosso caso os beans são os visitors.\\
		\lstset{language=XML}
		\begin{lstlisting}
		<bean id="tsVisitor" 
		     class="br.unb.cic.sa.visitors.TryStatementVisitor"/>
		\end{lstlisting}
	\end{frame}	
	
	
	\begin{frame}[fragile]\frametitle{XML}
		É quem possui os beans e onde serão injetados cada um desses beans.\\
		\lstset{language=XML}
		\begin{lstlisting}
		 <bean id="pa" class="br.unb.cic.sa.ProjectAnalyser">
			     atributo listVisitors
		     <property name="listVisitors">
		         pode ser uma listas ou outro obj qualquer, um Bean
		     </property>
		 </bean>
		\end{lstlisting}
	\end{frame}	
	
	
	\begin{frame}[fragile]\frametitle{Código}
		\centering
		\begin{verbatim}
			ApplicationContext ctx = 
			     	new ClassPathXmlApplicationContext("Beans.xml");
		\end{verbatim}
		Convém que seja tranformado em um sigleton.
		
	\end{frame}

	\begin{frame}\frametitle{Referências}
		\centering
		\begin{itemize}
			\item http://spring.io/docs
			\item http://stackoverflow.com/
		\end{itemize}
	\end{frame}
\end{document}