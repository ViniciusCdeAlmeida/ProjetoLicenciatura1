\documentclass[]{beamer}
\mode<presentation> {
	
	% The Beamer class comes with a number of default slide themes
	% which change the colors and layouts of slides. Below this is a list
	% of all the themes, uncomment each in turn to see what they look like.
	
	%\usetheme{default}
	%\usetheme{AnnArbor}
	%\usetheme{Antibes}
	%\usetheme{Bergen}
	\usetheme{Berkeley}
	%\usetheme{Berlin}
	%\usetheme{Boadilla}
	%\usetheme{CambridgeUS}
	%\usetheme{Copenhagen}
	%\usetheme{Darmstadt}
	%\usetheme{Dresden}
	%\usetheme{Frankfurt}
	%\usetheme{Goettingen}
	%\usetheme{Hannover}
	%\usetheme{Ilmenau}
	%\usetheme{JuanLesPins}
	%\usetheme{Luebeck}
	%\usetheme{Madrid}
	%\usetheme{Malmoe}
	%\usetheme{Marburg}
	%\usetheme{Montpellier}
	%\usetheme{PaloAlto}
	%\usetheme{Pittsburgh}
	%\usetheme{Rochester}
	%\usetheme{Singapore}
	%\usetheme{Szeged}
	%\usetheme{Warsaw}
	
	% As well as themes, the Beamer class has a number of color themes
	% for any slide theme. Uncomment each of these in turn to see how it
	% changes the colors of your current slide theme.
	
	%\usecolortheme{albatross}
	%\usecolortheme{beaver}
	%\usecolortheme{beetle}
	%\usecolortheme{crane}
	%\usecolortheme{dolphin}
	%\usecolortheme{dove}
	%\usecolortheme{fly}
	%\usecolortheme{lily}
	%\usecolortheme{orchid}
	%\usecolortheme{rose}
	%\usecolortheme{seagull}
	%\usecolortheme{seahorse}
	%\usecolortheme{whale}
	%\usecolortheme{wolverine}
	
	%\setbeamertemplate{footline} % To remove the footer line in all slides uncomment this line
	%\setbeamertemplate{footline}[page number] % To replace the footer line in all slides with a simple slide count uncomment this line
	
	%\setbeamertemplate{navigation symbols}{} % To remove the navigation symbols from the bottom of all slides uncomment this line
}

\usepackage[utf8]{inputenc} % codificacao de caracteres
\usepackage[T1]{fontenc}    % codificacao de fontes
\usepackage[brazil]{babel}  % idioma
\usepackage{listings}
\usepackage{color}
\usefonttheme[onlymath]{serif} % fonte modo matematic
\usepackage{adjustbox}
\usepackage{ragged2e}
\definecolor{gray}{rgb}{0.4,0.4,0.4}
\definecolor{darkblue}{rgb}{0.0,0.0,0.6}
\definecolor{cyan}{rgb}{0.0,0.6,0.6}

\lstset{
	basicstyle=\ttfamily,
	columns=fullflexible,
	showstringspaces=false,
	commentstyle=\color{gray}\upshape
}

\lstdefinelanguage{XML}
{
	morestring=[b]",
	morestring=[s]{>}{<},
	morecomment=[s]{<?}{?>},
	stringstyle=\color{black},
	identifierstyle=\color{darkblue},
	keywordstyle=\color{cyan},
	morekeywords={xmlns,version,type}% list your attributes here
}


%Titulo
\title[\sc{Static Analisys}]{\textbf{Static analysis to detect evolution language in java projects}}
\author[Thiago Cavalcanti - Vinícius Correa]{Thiago Gomes Cavalcanti \\ Vinícius Correa de Almeida}

\institute{
	\includegraphics[scale=0.6]{as_comp_cor.eps}\\
	\textbf{Orientador: Prof. Dr. Rodrigo Bonifácio de Almeida}
}
\date[28/06/2015]{28 de junho de 2015}

\begin{document}
	\section{Início}
	\begin{frame}
		\titlepage
	\end{frame}
	
	\section{Roteiro}
	\begin{frame}
		\frametitle{Roteiro}
		\tableofcontents[pausesections]
	\end{frame}

	
%%%%%%%%%%%%%%%%%%%%%%%%%%% CENÁRIO %%%%%%%%%%%%%%%%%%%%%%%%%%%
	\section{Cenário}
	\begin{frame}[label=problems]
		\frametitle{O Software}
		\begin{itemize}
			\item Software novo com construções ultrapassadas.
			\item Código de baixa qualidade e difícil entendimento.
			\item Dificuldade em efetuar refactoring e testes.
			\item Falhas de segurança inerentes ao desenvolvedor.
			\item Desenvolvedores não conhecem de novas features da linguagem.
			\item Time que está ganhando não se mexe!!!
		\end{itemize}
	\end{frame}
	
	
	\begin{frame}[fragile, t]
		\frametitle{Cenário}
		\centering \textbf{Em 2004, fora introduzido \textit{Generics} em Java5.}
		\begin{table}[ht]
			\centering
			\adjustbox{max height=\dimexpr\textheight-5.0cm\relax,
				max width=\textwidth}{
				\begin{tabular}{c|c}% centered column
					\hline \hline
						\textbf{Type}               &   \textbf{Parameterizations}\\ \hline
						List<String> 				&	1066 \\ \hline
						ArrayList<String> 			&	682 \\ \hline
						HashMap<String,String> 		&	554 \\ \hline
						List<ObjectTreeNode> 		&	376 \\ \hline
						List<ITableInfo> 			&	322 \\ \hline
						Class<?> 					&	314 \\ \hline
						List<TableColumnInfo> 		&	304 \\ \hline
						Vector<String> 				&	234 \\ \hline
						List<ArtifactStatus> 		&	196 \\ \hline
						Collection<String[]> 		&	166 \\ \hline
						List<Object[]> 				&	132 \\ \hline
						Iterator<String> 			&	124 \\ \hline
						ArrayList<MappedClassInfo> 	&	114 \\ \hline
						Set<String> 				&	102 \\ \hline
				\end{tabular}
			}
			\caption{Em 2011 +500Milhões loc em 20 projetos por Chris Parnin.}	
		\end{table}
	\end{frame}	

%%%%%%%%%%%%%%%%%%%%%%%%%%% CENÁRIO %%%%%%%%%%%%%%%%%%%%%%%%%%%



%%%%%%%%%%%%%%%%%%%%%%%%%%% MOTIVAÇÃO %%%%%%%%%%%%%%%%%%%%%%%%%%%
	\section{Motivação}
	
	\begin{frame}[label=Motivação]
		\frametitle{Motivação}
		\begin{block}{Fortran}
			\begin{itemize}
				\item Remoção do ultrapassado estilo \textit{do loop} caso este terminasse com \textit{continue}, era substituído por construção equivalente \textit{end do}.
				
				\item Remoção do \textit{goto} por uma construção \textit{case} equivalente.
				
				\item Uso de palavras reservadas \textit{if, while, ...}, como variáveis.
				
				\item Introdução de OO em Fortran 2003 e um formato de fonte projetada para 80 colunas de cartão perfurados. Após essa versão os compiladores não forneciam suporte para versão anterior.
			\end{itemize}
		\end{block}
	\end{frame}
	
	\begin{frame}[label=AspectosAnalisados]
		\frametitle{Aspectos analisados}
		\begin{block}{Evoluções em Java}
			\begin{itemize}
				\item Novas características introduzidas:
					\begin{itemize}
						\item Como a comunidade reage a novas características.
						\item São utilizadas por poucos.
						\item IDEs fornecem suporte.
					\end{itemize}
					
				\item Identificas quais novas características são mais usadas.
				\item Foco no software livre desenvolvido em JAVA.
				\item Falta de suporte da IDE.
			\end{itemize}
		\end{block}
	\end{frame}
	
	\begin{frame}[label=ApectosAnalisados]
		\frametitle{Aspectos analisados}
		
			\begin{block}{Oportunidade de evolução em algumas características como:}
				\begin{itemize}
					\item Multicatch, \textit{catch(E1 | E2 | ...)}.
					\item Try com Resources, \textit{try(BufferReader obj)}.
					\item LambdaFunction, \textit{loop com condicional}.
					\item Switch(string), \textit{if(string).. elseif(...) aninhados}.
					\item Métodos com vargs, \textit{methodName(String...)}.
					\item Usar API Stream em collections para trabalhar de forma Funcional.
				\end{itemize}
			\end{block}
	\end{frame}
	

	
%	\begin{frame}[label=DimensaoTrabalhos]
%		\frametitle{Dimensões Alcancáveis}
%		\begin{itemize}
%			\item Falar da pretenção do analisador estático reduzir custo no desenvolvimento.
%			\item E de como pode ajudar a ORACLE a tornar mais limpa as releases.
%			
%		\end{itemize}
%	\end{frame}
%%%%%%%%%%%%%%%%%%%%%%%%%%% MOTIVAÇÃO %%%%%%%%%%%%%%%%%%%%%%%%%%%	



%%%%%%%%%%%%%%%%%%%%%%%% Revisão teórica %%%%%%%%%%%%%%%%%%%%%%%%
	\section{Revisão Teórica}
	\frametitle{Referências}
	\begin{frame}[label=referencias]
		\begin{block}{Referências}
			\begin{itemize}
				\item Chris Parnin, Christian Bird, and Emerson Murphy-Hill 2011. \textbf{Java generics adoption: How new features are introduced, championed, or ignored}.
				
				\item Jefrey L. Overbey and Ralph E. Johnson. \textbf{Regrowing a language: refactoring tools
				allow programming languages to evolve}.
				
				\item Dyer, Robert and Rajan, Hridesh and Nguyen, Hoan Anh and Nguyen, Tien N. \textbf{Mining billions of AST nodes to study actual and potential usage of Java language features}.
			\end{itemize}
			
		\end{block}
	\end{frame}
	
%%%%%%%%%%%%%%%%%%%%%%%% Revisão teórica %%%%%%%%%%%%%%%%%%%%%%%%



%%%%%%%%%%%%%%%%%%%%% Diretrizes da pesquisa %%%%%%%%%%%%%%%%%%%%
	\section{Diretrizes da pesquisa}

	\begin{frame}[fragile]\frametitle{Problema}
		\begin{block}{Problema}
			Quais as características da linguagem Java são mais utilizadas no desenvolvimento do software e como é a adoção de novas características da linguagem através das versões dos softwares?
		\end{block}
	\end{frame}	
	
	
	\begin{frame}[fragile]\frametitle{Hipóteses}
		
	%	\begin{description}\item[\textbf{Hipótese 1}]\end{description}
		\begin{block}{Hipótese 1}
			\begin{itemize}
				\item Novas características são pouco utilizadas ou ignoradas no desenvolvimento do software.
			\end{itemize}
		
		\end{block}
		
		%\begin{description}\item[\textbf{Hipótese 2}]\end{description}
		\begin{block}{Hipótese 2}
			\begin{itemize}
				\item Novas Códigos obsoletos são mantido em todas as releases do software.
			\end{itemize}
		\end{block}
		
	%	\begin{description}	\item[\textbf{Hipótese 3}]\end{description}
		\begin{block}{Hipótese 3}
			\begin{itemize}
				\item Versões mais recentes do software não possem as últimas características adotadas na linguagem.
			\end{itemize}
		\end{block}
		
	\end{frame}
	
	
	\begin{frame}[fragile]\frametitle{Objetivos}
		%\begin{description}	\item[\textbf{Objetivo Geral}]\end{description}
		\begin{block}{Geral}
			\begin{itemize}
				\item Identificar de forma eficiente código obsoleto para cada versão ao qual o software foi lançado.
			\end{itemize}
		\end{block}
			
		%\begin{description}	\item[\textbf{Objetivo Geral}]\end{description}
			\begin{block}{Específicos}
				\begin{itemize}
					\item Elaborar um analisador estático que localize possíveis casos de código obsoleto.
					\item Exibir um relatório final indicando os possíveis casos e a possível evolução para tal código.
				\end{itemize}
			\end{block}
	\end{frame}
%%%%%%%%%%%%%%%%%%%%% Diretrizes da pesquisa %%%%%%%%%%%%%%%%%%%%%
	
% % % % % % % % % % % % % % % Metodologia % % % % % % % % % % % % % % % % % %	
	\section{Metodologia}
	\begin{frame}[label=metodologia, fragile]
		\frametitle{Procedimentos}
		\begin{block}{Ordem das tarefas}
			\begin{itemize}
				\item Revisão Bibliográfica
					\begin{itemize}
						\item Seleção do projetos.
						\item Escolha de projetos opensource.
					\end{itemize}
				\item Implementação do analisador estático.
					\begin{itemize}
						\item Definir arquitetura.
						\item Implementação.
						\item Coleta de dados.
					\end{itemize}
				\item Refazer os passos do artigo de Generics(escrever artigo).
				\item Análise do dados coletados utilizando R.
				\item Defesa TCC.
			\end{itemize}
		\end{block}
	\end{frame}
	
	\begin{frame}[fragile]
		\frametitle{Cronograma}
		\begin{block}{Calendário}
			\begin{table}[ht]
				\centering
				\resizebox{\textwidth}{!}{
					\begin{tabular}{c|c|c|c|c|c|c|c|c|c}% centered column
						\hline \hline
						\textbf{Atividades}   & ABR & MAI & JUN & JUL & AGO & SET & OUT & NOV & DEZ\\ \hline
						Revisão bibliográfica & X & X & X &   &   &   &   &   & \\ \hline
						Implementação         &   & X & X & X & X &   &   &   & \\ \hline
						Coleta de Dados       &   &   & X & X & X & X & X &   & \\ \hline
						Escrever artigo       &   &   & X & X &   &   &   &   & \\ \hline
						Análise de dados      &   &   &   & X & X &   &   & X & \\ \hline
						Defesa TCC            &   &   &   &   &   &   &   &   & X\\ \hline
					\end{tabular}
				}
				\caption{Cronograma das atividades}
			\end{table}
		\end{block}
		
	\end{frame}
% % % % % % % % % % % % % % % Metodologia % % % % % % % % % % % % % % % % % %		
	

%%%%%%%%%%%%%%%%%%%%% Resultados esperados %%%%%%%%%%%%%%%%%%%%%
	\section{Resultados esperados}
	\begin{frame}[fragile, label=re]\frametitle{Resultados esperados}
		\begin{block}{Resultados}
			\begin{itemize}
				\item Desenvolver uma ferramenta capaz analisar estaticamente um projeto Java e encontrar construções ultrapassadas.
				
				\item Propor um relatório contendo a localização do código ultrapassado com sua possível atualização, cabendo ao desenvolvedor escolher ou não.
				
				\item Substituir código ultrapassado por um mais atual.(Futuro)
			\end{itemize}
		\end{block}
	\end{frame}
%%%%%%%%%%%%%%%%%%%%% Resultados esperados %%%%%%%%%%%%%%%%%%%%%


%%%%%%%%%%%%%%%%%%%%% Encerramento %%%%%%%%%%%%%%%%%%%%%
	\section{Encerramento}
	\begin{frame}[label=encerramento]
		\frametitle{Encerramento}
		\begin{block}{Perguntas}
			\centering
			\includegraphics[scale=0.6]{duvidas.jpg}\\
		\end{block}
	\end{frame}
%%%%%%%%%%%%%%%%%%%%% Encerramento %%%%%%%%%%%%%%%%%%%%%

\end{document}