\chapter{Introdução}
\section{Introdução}


Com o passar do tempo as linguagens de programação evoluem entretanto não sabe-se ao certo como os softwares projetados e implementados há alguns anos acompanharam tais atualizações\cite{dyer2013large}. Conforme explicado por \cite{Overbey:2009:RLR:1639949.1640127}, tal evolução faz com que características obsoletas sejam mantidas e raramente são removidas de uma linguagem o que acarreta em um aumento da complexidade, aprendizagem e da manutenção do software. Isso naturalmente aumenta a dificuldade de desenvolvimento o que resulta em um aumento de dificuldade de aprendizagem de determinada versão já ultrapassada de uma linguagem e faz com que a equipe alterne entre propriedades atuais e antigas as quais passam a ser quase um dialeto da linguagem implicando no aumento de tempo para conceber um projeto e consequentemente gerindo aumento no custo final projeto.\\

Uma decisão não é tão simples de manter uma porção do código congelado, sem evolução, ao longo projeto devido alguma restrição técnica. O que infelizmente acarreta em uma estagnação de todo um sistema pois não é somente o projeto afetado, mas sim uma toda infraestrutura como compiladores, banco de dados e sistema operacional e que se de alguma forma vierem a ser atualizados com esta porção código estagnado pode ocasionar problemas como uma queda significativa de desempenho ou até mesmo o sistema parar de funcionar. Devido a esses problemas de código não atualizado com as versões com estruturas mais atuais a proposta da realização de refatoração através de ferramenta a serem desenvolvidas que visem atacar esse gargalo deixado por código obsoleto.\\

A base para tal trabalho será o desenvolvimento de algumas ferramentas que auxiliem em mudanças em um código legado para reduzir suas estruturas obsoletas e com baixa performance. Essas ferramentas tem como base a sua construção em linguagem {\it Java} com o intuito de tornar o processo de construção mais ágil e posteriormente aberto para o acoplamento de novos módulos. A construção de uma árvore sintática é um passo onde é feito para cada arquivo de código {\it .Java} e posteriormente de todos os arquivos {\it .Java} contidos em qualquer projeto para posterior análise. Este parser implica em listar todos os arquivos Java e gerar uma Árvore Sintática Abstrata (AST) para que posteriormente haja um percorrimento usando um visitor\cite{Gamma:1995:DPE:186897} nos blocos de código contidos nos nós da (AST)\cite{Dyer:2014:MBA:2568225.2568295} afim de compará-los como estes com a versão atual.\\

O processo de refatoração\cite{Schaefer:2010:SIR:1932682.1869485} tem como motivação a reestruturação de código, de forma que o código considerado pelo processo, morto,duplicado ou com perca de desempenho não haja código morto, duplicado ou com perca de desempenho em um determinado trecho de código. Esse processo não tem como premissa a atualização do código para novas estruturas de versões da linguagem mais recentes. Essa nova abordagem de refatoração tem com motivação a retirada de código obsoleto, devido a novas abordagens e estruturas das novas versões, e com baixo desempenho de sistemas sem prejuízo na sua engenharia e funcionamento.\\

Devido a esse tipo de refatoração de código visa a evolução do código para a uma mais recente com estruturas onde não haja perca de desempenho devido a mudança e também a atualização do código legado para estruturas modernas. Tais estruturas antigas com a ação dessa proposta de refatoração tendem com o tempo sair das estruturas providas pela linguagem Java como aconteceu com Fortan 90 como visto em \cite{Overbey:2009:RLR:1639949.1640127} e essa abordagem tem o intuito de diminuir a quantidade de estruturas antigas que não fazem mais sentido pois novas estruturas realizam as mesmas funções no interpretador da linguagem Java. As modificações proposta pela ferramenta de refatoração não deverá trazer com que haja perca de desempenho ou aumento da complexidade do código portanto deixando como uma sugestão a alteração do código ou segmento de código para o usuário a adesão das propostas de refatoração ou não.\\

A árvore de sintaxe abstrata (AST) proposta por Chomsky em 1956\cite{chomsky1956three},  é uma estrutura de dados que representa estruturas de cadeias sintáticas onde é reapresentada por um esqueleto semântico da linguagem em questão. É constituída através de um framework do ambiente de desenvolvimento integrado chamado Eclipse onde o nome desse framework é chamado de EclispeJDT onde traz métodos implementados pelo próprio framework para o percorrimento e ações na árvore sintática. A ideia é transformar inicialmente qualquer código fonte java em uma árvore sintática e devido a isso, o mapeamento da arvore já construída que é muito conveniente para inspecionar o código fonte de um arquivo ou de um projeto com vários arquivos em diferentes diretórios. Com isso é possível realizar ou sugerir ao usuário modificações no código fonte através desta árvore construída e isto seria referenciado automaticamente no código fonte.\\

A proposta é criar ferramentas de análise estática para códigos fonte da linguagem Java para que possa-se apurar projetos pequenos e posteriormente em projetos {\it open-sources} para a verificação da existência de alguma defasagem\cite{dyer2013large} de estruturas entre qualquer versão da linguagem Java que fora concebido para a versão atual e estável na qual a linguagem se encontra. O desenvolvimento das ferramentas será com a versão mais atual da linguagem Java que neste momento é o Java versão 8 para verificar se os softwares desenvolvidos está versão nesta versão e como acompanharam a evolução de décadas de novas versões sem atualização do código por motivo de engenharia outro qualquer.\\