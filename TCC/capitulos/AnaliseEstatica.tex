\section{Análise estática}

Análise estática é uma técnica automática usada no processo de verificação e que n\~{a}o requer a 
execu\c c\~{a}o do software. Para a linguagem Java, duas estrat\'{e}gias s\~{a}o tipicamentes usadas pelas ferramentas. 
A primeira utiliza basicamente o código fonte dos programas nas an\'{a}lises, enquanto que a segunda estrat\'{e}gia 
realiza as an\'{a}lises usando o c\'{o}digo compilado dos programas (i.e., o \emph{bytecode} correspondente)~\cite{Ayewah:2008:USA:1439186.1439221}. 
Neste trabalho, o analisador est\'{a}tico segue a primeira estrat\'{e}gia, realizando as an\'{a}lises 
usando exclusivamente o código fonte. % sem que tenha sido executado devido a flexibilidade e infraestrutura consolidada encontrada no eclipse AST.
Um fato importante é que a análise est\'{a}tica somente obtém sucesso quando s\~{a}o identificados determinados padrões de uso 
de constru\c c\~{o}es de software. Neste projeto, tais padr\~{o}es são identificados por {\it visitors}~\cite{Gamma:1995}, implementados usando 
a infraesturua do Eclipse JDT (Java Development Tools).
Este esfor\c co de verificação de software permite detectar falhas em fases iniciais de  desenvolvimento, 
evitando que \emph{bugs} e falhas sejam introduzidas e até mesmo postergados. O {\it  feedback} rápido pode alertar as equipes sobre 
falhas ocorridas, sendo uma abordagem complementar a casos de testes---  a partir do momento que um analisador est\'{a}tico encontra 
uma poss\'{i}vel viola\c c\~{a}o, é possível criar um teste de caso para que esta seja testada; aumentando assim a confiabilidade do software.
Por outro lado, existem limitações nestes verificadores estáticos como em software desenvolvidos sem qualquer uso de padrões ou 
sem arquiteturas consolidadas, criado por equipes composta de desenvolvedores inexperientes o qual a ferramente poderá apontar erros que são falsos positivos. 
Ou seja, desvios detectados que não correspondem a erros. 
Tais \emph{falsos positivos} são desagradáveis, porém não comprometem, a princ\'{i}pio, a qualidade do 
software sendo desenvolvido. Por outro lado, a ocorr\^{e}ncia de falsos positivos pode afetar outras atividades, como a de {\it refactoring}, 
uma vez que dificulta a melhoraria de um código que não segue padrão. Vale ainda ressaltar que a penalidade de encontrar um falso 
positivo envolve a perda de tempo em fazer uma inspeção no código para comprovar se é ou não um desvio de qualidade. 
Também há a possibilidade de falsos negativos, o que cabe ao programador verificar para evitar que tais limitação do analisador não se propague durante o ciclo de desenvolvimento.
