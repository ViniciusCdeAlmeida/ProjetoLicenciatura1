\chapter{Obtenção de Dados}

Para executar o analisador é necessário um \acs{CSV} que possui em seus campos o nome dos projeto, versões e caminho para que cada um seja analisado. Após o {\it input} é realizado uma pesquisa dados o caminho fornecido pelo \acs{CSV}.
É elaborada um pesquisa pelos arquivos fontes \textit{Java} e são armazenados em uma \textit{Collection} para que seja efetuada uma contagem das \acs{LOC} com o programa \acs{CLOC}.

Posteriormente os arquivos armazenados na \textit{Collection} são direcionados para a classe \textit{ProjectAnaliser.java} que que esta gere um \textit{parser} para cada arquivo e os \textit{visitors} elaborem uma pesquisa conforme suas configurações. Cada característica pesquisada é armazenada em uma \textit{Collection} de resultado a qual ao final de todo processo será tratada e cada característica será escrita um um arquivo \acs{CSV}.


\section{Tratar Dados}
Os dados obtidos serão tratados com o software R \cite{R} versão (3.1.2) além da biblioteca ggplot \cite{ggplot} versão (1.0.1) para gerar gráficos mais elegantes.


\section{Projetos Escolhidos}
Os projetos foram escolhidos baseados nos estudos realizados por \cite{Parnin:2011:JGA:1985441.1985446} e \cite{Dyer:2014:MBA:2568225.2568295}. Este projetos totalizaram um total de mais de 67M \acs{LOC} onde a tabela \ref{tab:ProjetosEscolhidos} exibe em detalhes as versões escolhidas.


\begin{table}[ht]
	\centering
	\caption{Projetos Escolhidos}
	\label{tab:ProjetosEscolhidos}
	\begin{tabular}{c|c}
		\hline \hline
		\textbf{Projeto}    &   \textbf{Versão}\\ \
	      Ant & 2 \\
	      Checkstyle &  \\
	      CommonsCollections &  \\
	      FindBugs &  \\
	      Hibernate & \\
	      JBoss & \\
	      Jetty & \\
	      Log4j & \\
	      Spring & \\
	      SquirrelSql & \\
	      Weka & \\
      \hline
    \end{tabular}
\end{table}