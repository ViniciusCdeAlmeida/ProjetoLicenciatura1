\chapter{Fundamentação}
\section{Linguagem Java}
No começo da década de 90 um pequeno grupo de engenheiros da Oracle chamados de "\textit{Green Team}" acreditava que a próxima onde de na area da computação seria a união de equipamentos eletroeletrônicos com os computadores. O "\textit{Green Team}" liderado por James Gosling, demonstraram que a linguagem de programação Java, que foi desenvolvida pela equipe e originalmente era chamado de Oak, foi desenvolvida para dispositivos de entretenimento como aparelhos de tv a cabo, porem não foi bem aceita no meio. Em 1995 com a massificação da Internet, a linguagem Java teve sua primeira grande aplicação o navegador Netscape.

Java é uma linguagem de programação de propósito geral orientada a objetos, concebida especificadamente para ter poucas dependências de implementação que isso acarreta que uma vez que a aplicação fora desenvolvida ela poderá ser executada em qualquer ambiente computacional.

Na sua primeira versão chamada de Java 1 (\acs{JDK} 1.0.2) haviam oito pacotes básicos do java como: java.lang, java.io, java.util, java.net, java.awt, java.awt.image, java.awt.peer e java.applet. Foi usado para o desenvolvimento de ferramentas populares na época como o Netscape 3.0 e o Internet Explorer 3.0.

Sua segunda versão foi o \acs{JDK}1.1 \cite{JDK1.1} que trouxe ganhos em funcionalidades, desempenho e qualidade. Novas aplicações tambem surgiram como : JavaBeans, aprimoramento do \acs{AWT}, novas funcionalidades como o \acs{JDBC}, acesso remoto ao objeto \acs{RMI} e suporte ao padrão Unicode 2.0.\\

A terceira versão Java 2 (\acs{JDK} 1.2) ofereceu melhorias significativas no desempenho, um novo modelo de segurança, flexível e um conjunto completo de aplicações de programação interfaces \acs{API}'s. O modelo de "\textit{sandbox}" foi ampliado para dar aos desenvolvedores, usuários e administradores de sistema a opção de especificar e gerenciar um conjunto de políticas de segurança flexíveis que governam as ações de uma aplicação ou \textit{applet} que pode ou não ser executada. Foi introduzido o suporte nativo a \textit{thread} para o ambiente operacional Solaris e também a compressão de memória para classes carregadas. Alocação de memória com mais desempenho e melhor para a coleta de lixo. Arquitetura de máquina virtual conectável para outras máquinas virtuais, incluindo a \textit{Java HotSpot VMNew}. \textit{Java Native Interface }\acs{JNI} de conversão. O novo conjunto de componentes de projeto, \acs{GUI} (\textit{Swing}). \acs{API} Java 2D que fornece novos recursos gráficos 2D e \acs{AWT}, bem como suporte para impressão. Java \textit{plug-in} para navegadores da \textit{web}, incluída na plataforma Java 2, fornecendo um tempo de execução totalmente compatível com a máquina virtual Java amplamente implantadas em navegadores. A inclusão do \acs{JDBC} que fornece um acesso mais fácil aos dados para consultas mais flexíveis e trouxe melhor desempenho e estabilidade promovidos por cursores de rolagem e suporte para SQL3 de tipos.\\


Em 8 de Maio de 2000 foi anunciado o Java 2 versão 1.3 que trouxe ganho de desempenho em relação a primeira versão da JS2E de cerca de 40\%  no tempo de {\it  start-up}. O {\it Java HotSpot VM} de cliente e suas bibliotecas atentando ao desempenho ao fazer o J2SE versão 1.3 a {\it realease} o mais rápido até à data. Novos recursos, como o {\it caching applet} e instalação do pacote opcional Java através da tecnologia Java {\it  Plug-in} para aumentar a velocidade e a flexibilidade com que os {\it applets} e aplicativos baseados na tecnologia Java podem ser implantados. Java {\it  Plug-in} tecnologia é um componente do ambiente de execução Java 2 que permite Java {\it applets} e aplicativos para a execução. O novo suporte para \acs{RSA} assinatura eletrônica, gerenciamento de confiança dinâmico, certificados X.509, e verificação de arquivos o que significa aumento das possibilidades que os desenvolvedores tem para proteger dados eletrônicos. Uma série de novos recursos e ferramentas de desenvolvimento da tecnologia J2SE versão 1.3 que permite o desenvolvimento mais fácil e rápido de aplicações baseadas na tecnologia {\it web} ou Java {\it  standalone} de alto desempenho. A adição de RMI/IIOP e o JNDI para a versão 1.3, melhora na interoperabilidade J2SE. Melhora da conectividade com sistemas de {\it  back-end} que suportam CORBA. O novo suporte que o JNDI fornece acesso aos diretórios que suportam o populares LDAP Lightweight Directory Access Protocol, entre outros.\\


No ano de 2002 no dia 6 de Fevereiro, foi lançado a J2SE versão 1.4. Com a versão 1.4, as empresas puderam usar a tecnologia Java para desenvolver aplicativos de negócios mais exigentes e com menos esforço e em menos tempo. As novas funcionalidades como a nova I/O e suporte a 64 bits. A J2SE se tornou plataforma ideal para a mineração em grande escala de dados, inteligência de negócios, engenharia e científicos. A versão 1.4 forneceu suporte aprimorado para tecnologias padrões da indústria, tais como SSL, LDAP e CORBA a fim de garantir a operacionalidade em plataformas heterogêneas, sistemas e ambientes. Com o apoio embutido para XML, a autenticação avançada, e um conjunto completo de serviços de segurança, está versão forneceu base para padrões de aplicações Web e serviços interoperáveis. O J2SE avançou para o desenvolvimento de aplicativos de cliente com novos controles de GUI, acelerou Java 2D, a performance gráfica, internacionalização e localização expandida de apoio, novas opções de implantação e suporte expandido para o até então {\it Windows XP}.\\

Com a chegada da \acs{JSE2} versão 1.5 (Java 5.0) em 30 de Setembro de 2004, impulsionou benefícios extensivos para desenvolvedores, incluindo a facilidade de uso, desempenho global e escalabilidade, monitoramento do sistema e gestão e desenvolvimento. O Java 5 foi derivado do trabalho de 15 componentes Java Specification Requests (JSRs) englobando recursos avançados para a linguagem e plataforma. Os líderes da indústria na época que participam no grupo de peritos J2SE 5.0 incluiram: Apache Software Foundation, Apple Computer, BEA Systems, Borland Software Corporation, Cisco Systems, Fujitsu Limited, HP, IBM, Macromedia, Nokia Corporation, Oracle, SAP AG, SAS Institute, SavaJe Technologies e Sun Microsystems.

Houve novas funcionalidades foram implementadas como: facilidade de desenvolvimento onde os programadores da linguagem Java pode ser mais eficiente e produtivos com os recursos de linguagem Java 5 que permitiram a codificação mais segura. Nesta versão surgiu o {\it Generics} ~\cite{OracleGenerics, bracha1998gj}, tipos enumerados, metadados e autoboxing de tipos primitivos permitindo assim uma fácil e rápida codificação. Monitoramento e gestão permitindo um foco para a nova versão da plataforma, a aplicativos baseados na tecnologia Java {\it Virtual Machine} que passou a ser monitorado e gerenciado com o {\it built-in} de suporte para Java {\it Management Extensions}. Isso ajudou a garantir que seus funcionários, sistemas de parceiros do cliente permanecessem em funcionamento por mais tempo. Suporte para sistemas de gestão empresarial baseados em SNMP também é viável. Um olhar novo sobre aplicativos, mais moderna, baseada na tecnologia Java padrão e proporciona uma sensação GUI para aplicativos baseados na tecnologia Java. A J2SE 5.0 teve suporte completo a internacionalização e também possuindo suporte para aceleração de hardware por meio da {\it API OpenGL } e também para o sistema operacional Solaris e sistemas operacionais da distribuição Linux. **Maior desempenho e escalabilidade com a nova versão que incluiu melhorias de desempenho, tais como menor tempo de inicialização, um menor consumo de memória e JVM auto ajustável para gerar maior desempenho geral do aplicativo e desenvolvimento em J2SE 5.0 em relação às versões anteriores.\\


Java 1.6 (Java 6) foi divulgado em 11 de dezembro de 2006. Tornou o desenvolvimento mais fácil, rápido e eficiente em termos de custos e ofereceu funcionalidades para serviços web, suporte linguagem dinâmica, diagnósticos e aplicações {\it desktop}. Com a chegada dessa nova versão do Java houve combinação com o {\it NetBeans} IDE 5.5 fornecendo aos desenvolvedores uma estrutura confiável, de código aberto e compatível, de alta performance para entregar aplicativos baseados na tecnologia Java mais rápido e fácil do que nunca. O {\it NetBeans} IDE fornece uma fonte aberta e de alto desempenho, modular, extensível, multiplataforma Java IDE para acelerar o desenvolvimento de aplicações baseadas em software e serviços {\it web}. O Java 1.6 ajudou a acelerar a inovação para o desenvolvedor, aplicativos de colaboração {\it online} e baseadas na {\it web}, incluindo um novo quadro de desenvolvedores \acs{API}'s para permitir a mistura da tecnologia Java com linguagens de tipagem dinâmica, tais como PHP, Python, Ruby e tecnologia JavaScript. A Sun também criou uma coleção de mecanismos de {\it script} e pré-configurado o motor {\it JavaScript Rhino} na plataforma Java. Além disso, o software inclui uma pilha completa de clientes de serviços web e suporta as mais recentes especificações de serviços {\it web}, como \acs{JAX-WS} 2.0, \acs{JAXB} 2.0, \acs{STAX} e \acs{JAXP}. A plataforma Java 1.6 forneceu ferramentas expandidas para o diagnóstico, gestão e monitoramento de aplicações e também inclui suporte para o novo {\it NetBeans Profiler} 5.5 para {\it Solaris DTrace } e, uma estrutura de rastreamento dinâmico abrangente que está incluído no sistema operacional Solaris 10. Além disso, o software Java SE 6 aumenta ainda mais a facilidade de desenvolvimento com atualizações de interface ferramenta para o {\it Java Virtual Machine} (\acs{JVM}) e o {\it Java Platform Debugger Architecture} (\acs{ACDP}).\\


Java 7 ~\cite{JSE7} foi lançado no dia 28 de julho de 2011. Essa versão foi resultado do desenvolvimento de toda a indústria envolvendo uma revisão de codigo aberto e extensa colaboração entre os engenheiros da {\it Oracle} e membros do ecossistema Java em todo o mundo através da comunidade {\it OpenJDK} e do {\it Java Community Process} (\acs{JCP}). Compatibilidade com versões anteriores de Java 7 com versões anteriores da plataforma a fim de preservar os conjuntos de habilidades dos desenvolvedores de software Java e proteger os investimentos em tecnologia Java. As alterações de linguagem ajudaram a aumentar a produtividade do desenvolvedor e simplificar tarefas comuns de programação, reduzindo a quantidade de código necessário, esclarecendo sintaxe e tornar o código com mais legibilidade. Melhor suporte para linguagens dinâmicas incluindo: Ruby, Python e JavaScript, resultando em aumentos substanciais de desempenho no \acs{JVM}. Uma nova API {\it multicore ready} que permite aos desenvolvedores para se decompor mais facilmente problemas em tarefas que podem ser executadas em paralelo em números arbitrários de núcleos de processador. Uma interface de I/O abrangente para trabalhar com sistemas de arquivos que podem acessar uma ampla gama de atributos de arquivos e oferecem mais informações quando ocorrem erros. Novos recursos de rede e de segurança. Suporte expandido para a internacionalização, incluindo suporte a Unicode 6.0. Versões atualizadas das bibliotecas padrão.\\

Com o lançamento do Java SE 8 em 18 de Março de 2014, permitiu uma maior produtividade e desenvolvimento de aplicativos significativos aumentos de desempenho através da redução de linhas de código, {\it collectons} melhoradas, modelos mais simples de programação paralela e uso mais eficiente de processadores {\it multi-core}. As principais características do \acs{JDK} 8 são o projeto Lambda, {\it Nashorn JavaScript Engine}, um conjunto de perfis compactas e a remoção da "geração permanente" do {\it HotSpot Java Virtual Machine} (\acs{JVM}). A \acs{JDK} 8 alcançou desempenho recorde mundial para 4 sistemas de soquete em servidores baseados em Intel e NEC por 2 sistemas de soquete em servidores SPARC da Oracle T5, com uma melhoria de desempenho de 12\% para 41\% em comparação com o JDK 7 na mesma configuração de Oracle.

O \acs{JDK} 8 adicionou novas funcionalidades como as expressões lambda são suportados pelas seguintes características: As referências a métodos são compactas, maior legibilidade expressões lambda para métodos que já têm um nome. Métodos padrão que permitem adicionar novas funcionalidades para as interfaces de suas bibliotecas e assegurar a compatibilidade binária com o código escrito para versões mais antigas dessas interfaces. Eles são os métodos de interface que têm uma aplicação e a palavra-chave padrão no início da assinatura do método. Além disso, pode-se definir métodos estáticos em interfaces. Novos e aprimorados APIs que se aproveitam de expressões lambda e dos {\it streams} em Java 8 descrevem as classes novos e aprimorados que se aproveitam de expressões lambda e {\it streams}. Aprimoramento do compilador Java aproveita digitação alvo para inferir os parâmetros de tipo de um método de invocação genérica. O tipo de destino de uma expressão é o tipo de dados que o compilador Java espera, dependendo de onde a expressão aparece, por exemplo, pode-se usar o tipo de destino de uma instrução de atribuição para o tipo de inferência em Java 7. No entanto, em Java 8, você pode usar o tipo de destino para a inferência de tipos em mais contextos. Anotações sobre tipos Java onde é possível aplicar uma anotação em qualquer lugar onde um tipo é usado, utilizado em um conjunto com um sistema de tipo de conector, isso permite a verificação de tipo mais forte de seu código e repetição de anotações que agora é possível aplicar o mesmo tipo de anotação mais de uma vez para a mesma declaração ou o tipo de utilização.


								