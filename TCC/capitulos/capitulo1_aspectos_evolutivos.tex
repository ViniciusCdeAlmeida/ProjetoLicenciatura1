\section {Aspectos evolutivos da liguagem Java}
		\subsection {Java 2}
			A primeira versão do Java Security, disponível no JDK 1.1 \cite{JDK1.1}, contém um subconjunto dessa funcionalidade, incluindo APIs para:
		  \begin{itemize}
			  \item Assinaturas Digitais: Algoritmos de assinatura digital, como DSA ou MD5 com RSA. A funcionalidade inclui a geração de chaves público/privado , bem como assinatura e verificação de dados digitais.
			  \item Gerenciamento de Chaves: Um conjunto de abstrações para o gerenciamento de ''diretores'' (entidades como usuários individuais ou grupos), suas chaves, e os seus certificados. Ele permite que aplicativos para projetar seu próprio sistema de gerenciamento de chaves, e para interoperar com outros sistemas em alto nível.
			  \item Lista de controle de acesso: Um conjunto de abstrações para o gerenciamento de ''diretores'' e suas permissões de acesso.
			  \item A obtenção de um objeto de assinatura: 
			  
\begin{lstlisting}
import java.security.Signature;
import java.security.NoSuchAlgorithmException;
	
public class SignFile {
	Signature signature;
		
	private void init(String algorithm) throws NoSuchAlgorithmException{
		signature = Signature.getSignature(algorithm);
    }
}
\end{lstlisting}
			  
			  \item Em versões anteriores, Java suportava apenas {\it top-level} classes, que devem ser membros de pacotes. Na versão 1.1, o programador Java pode agora definir classes internas como membros de outras classes \cite{bracha1998gj}, localmente dentro de um bloco de instruções, ou (anonimamente) dentro de uma expressão.
		  
\begin{lstlisting}
public class FixedStack {
	...
	 public java.util.Enumeration elements() {
	     return new FixedStack$Enumerator(this);
	 }
}
		
class FixedStack$Enumerator implements java.util.Enumeration {
	private FixedStack this$0;
	
	FixedStack$Enumerator(FixedStack this$0) {
		this.this$0 = this$0;
		this.count = this$0.top;
	 }
			
	int count;
	public boolean hasMoreElements() {
		return count > 0;
	}
		
	public Object nextElement() {
		if (count == 0)
			throw new NoSuchElementException("FixedStack");
		
		return this$0.array[--count];
	}
}
\end{lstlisting}
			
			\clearpage
			\item Para escrever um objeto remoto (RMI), você escrever uma classe que implementa uma ou mais interfaces remotas. 
			
\begin{lstlisting}
package examples.hello;
public interface Hello extends java.rmi.Remote {
	String sayHello() throws java.rmi.RemoteException;
}
\end{lstlisting}
	 
\item HelloImpl.java
\begin{lstlisting}
package examples.hello;

import java.rmi.;
import java.rmi.server.UnicastRemoteObject;

public class HelloImpl extends UnicastRemoteObject implements Hello{
	private String name;
	
	public HelloImpl(String s) throws RemoteException {
		super();
		name = s;
	}
	
	public String sayHello() throws RemoteException {
		return  "Hello World!";
	}
	
	public static void main(String args[]){
	
		System.setSecurityManager(new RMISecurityManager());
	
		try {
			HelloImpl obj = new HelloImpl("HelloServer");
			Naming.rebind("//myhost/HelloServer", obj);
			System.out.println("HelloServer bound in registry");
		} catch (Exception e) {
			System.out.println("HelloImpl err: " + e.getMessage());
			e.printStackTrace();
		}
	}
}
\end{lstlisting}
		\end{itemize}


	\clearpage
	\subsection {Java 4}
	  \begin{itemize}
		  \item {\it Assertion Facility} \cite{JSE8_Enhancements}. As {\it assertions} são expressões booleanas que o programador acredita ser verdade sobre o estado de um programa de computador. Por exemplo, depois de ordenar uma lista o programador pode afirmar que a lista está em ordem crescente. Avaliando as afirmações em tempo de execução para confirmar a sua validade é uma das ferramentas mais poderosas para melhorar a qualidade do código, uma vez que rapidamente se descobre equívocos do programador sobre o comportamento de um programa.
	  \end{itemize}
	
	\subsection {Java 5}
	  \begin{itemize}
		  \item {\it Generics}\cite{JSE8_Enhancements, OracleGenerics,Parnin:2011:JGA:1985441.1985446}. Este novo recurso para o sistema de tipo permite que um tipo ou método operar em objetos de vários tipos, proporcionando em tempo de compilação tipo de segurança. Acrescenta em tempo de compilação um tipo de segurança para as {\it collections} e elimina o trabalho penoso de {\it casting}. Um exemplo do uso de {\it colletions} e {\it generics} respectivamente:
\begin{lstlisting}
static void expurgate(Collection c) {
	for (Iterator i = c.iterator(); i.hasNext(); )
		if (((String) i.next()).length() == 4)
			i.remove();
	}
	
static void expurgate(Collection<String> c) {
	for (Iterator<String> i = c.iterator(); i.hasNext(); )
		if (i.next().length() == 4)
			i.remove();
}
\end{lstlisting}
		  
		\item {\it For-Each Loop}. Esta nova estrutura de linguagem elimina o trabalho e erro de propensão de iteradores e variáveis de índice quando a iteração ocorre sobre coleções e arrays. Como a construção evoluiu com o advento dessa nova estrutura:
	
\begin{lstlisting}
void cancelAll(Collection<TimerTask> c) {
	for (Iterator<TimerTask> i = c.iterator(); i.hasNext(); )
	   i.next().cancel();
}
	
void cancelAll(Collection<TimerTask> c) {
	for (TimerTask t : c)
		t.cancel();
}
\end{lstlisting}
	  
	  \clearpage
	  \item {\it Varargs}. Esta nova estrutura tende a eliminar a necessidade de passagem manual de listas de argumentos em um array ao invocar métodos que aceitam de um comprimento variável de uma lista de argumentos. Nas versões anteriores, um método levava um número arbitrário de valores necessários a  criar uma matriz e colocar os valores para a matriz antes de chamar o método.


\begin{lstlisting}
public class Test {
	public static void main(String[] args) {
	  int passed = 0;
	  int failed = 0;
	  for (String className : args) {
	      try {
	          Class c = Class.forName(className);
	          c.getMethod("test").invoke(c.newInstance());
	          passed++;
	      } catch (Exception ex) {
	          System.out.printf("%s failed: %s%n", className, ex);
	          failed++;
	      }
	  }
	  System.out.printf("passed=%d; failed=%d%n", passed, failed);
	}
}
\end{lstlisting}
 
	  \item {\it Autoboxing/Unboxing}. Esta nova estrutura elimina o trabalho de conversão manual entre tipos primitivos (como {\it int}) e os tipos de classes {\it wrapper}
  \end{itemize}
  
  
  
	\subsection {Java 6}
		Não ocorram mudanças ou introdução de novas estruturas na linguagem Java \cite{JSE8_Enhancements}.
	
	\subsection {Java 7}
		\begin{itemize}
		  \item {\it Multi Catch} e lançamento de exceções com melhora na verificação de tipos. Um único bloco {\it catch} poderá lidar com mais de um tipo de exceção. Além disso, o compilador executa a análise mais precisa das exceções. Isso permite que o programador especifique tipos de exceção mais específicos na cláusula de uma declaração método. Um exemplo de como era as estruturas que usavam {\it cacths} e com a introdução de {\it multi catch} com o Java 7 \cite{JSE7}, respectivamente.
  

\begin{lstlisting}
catch (IOException ex) {
	logger.log(ex);
	throw ex;
}catch (SQLException ex) {
	logger.log(ex);
	throw ex;
}
\end{lstlisting}
\clearpage
\begin{lstlisting}
catch (IOException|SQLException ex) {
	logger.log(ex);
	throw ex;
}
\end{lstlisting}


 \item O {\it try-with-resouces}. A declaração {\it try-with-resouces} é uma instrução {\it try} que declara um ou mais recursos. Um recurso é um objeto que deve ser fechada após o programa terminar com ele. Essa declaração garante que cada recurso é fechada no final da declaração\cite{JSE7_Advanced}.
	 
\begin{lstlisting}

public static void writeToFileZipFileContents(
			String zipFileName, String outputFileName) throws java.io.IOException {
	
	java.nio.charset.Charset charset = java.nio.charset.StandardCharsets.US_ASCII;
	java.nio.file.Path outputFilePath = java.nio.file.Paths.get(outputFileName);
	
	try(
		 java.util.zip.ZipFile zf = new java.util.zip.ZipFile(zipFileName);
		 java.io.BufferedWriter writer = java.nio.file.Files.newBufferedWriter(outputFilePath, charset)
	){
	
		for (java.util.Enumeration entries = zf.entries(); entries.hasMoreElements();) {
			 String newLine = System.getProperty("line.separator");
			 String zipEntryName = ((java.util.zip.ZipEntry)entries.nextElement()).getName() + newLine;
			 writer.write(zipEntryName, 0, zipEntryName.length());
		 }
	}
}

\end{lstlisting}
		  
		  \clearpage
		  \item Inferência de tipos para criação de instâncias em {\it generics}\cite{OracleGenerics}\cite{Bracha:1998:MFS:286942.286957}\cite{Parnin:2011:JGA:1985441.1985446}. Com o Java 7 pode-se substituir os argumentos de tipo necessários para invocar o construtor de uma classe genérica com um conjunto vazio de parâmetros de tipo (<>), desde que o compilador infira os argumentos de tipo a partir do contexto. Este par de colchetes angulares é informalmente chamado de diamante.
  
  

\begin{lstlisting}
Map<String, List<String>> myMap = new HashMap<String, List<String>>();
Map<String, List<String>> myMap = new HashMap<>();
	
List<String> list = new ArrayList<>();
list.add("A");

list.addAll(new ArrayList<>());
	
class MyClass<X> {
	<T> MyClass(T t) {
	...
	}
}
\end{lstlisting}
	 
	  \end{itemize}
	  
	  
	\subsection{Java 8}
	  \begin{itemize}
		  \item Melhoria na inferência de tipos. O compilador Java aproveita digitação para inferir os parâmetros de tipo de uma invocação de método genérica. O tipo de destino de uma expressão é o tipo de dados que o compilador Java espera, dependendo de onde a expressão aparece. Por exemplo, pode-se usar o tipo de destino de uma instrução de atribuição para o tipo de inferência em Java 7. No entanto, em Java 8, pode-se usar o tipo de destino para a inferência de tipos em mais contextos. O exemplo mais proeminente está usando tipos de destino de um método de invocação para inferir os tipos de dados dos seus argumentos.

\begin{lstlisting}
	List<String> stringList = new ArrayList<>();
	stringList.add("A");
	stringList.addAll(Arrays.asList());
\end{lstlisting}
		  
		  
		  \item Expressões lambda. Permitem encapsular uma única unidade de comportamento e passá-lo para outro código. Pode-se usar uma expressãos lambda, se quiser uma determinada ação executada em cada elemento de uma {\it collection}, quando o processo for concluído, ou quando um processo encontra um erro. \cite{JSE7} \\
	  \end{itemize}

\clearpage
\begin{lstlisting}
public class Calculator {
	
	interface IntegerMath {
		int operation(int a, int b);   
	}

	public int operateBinary(int a, int b, IntegerMath op) {
		return op.operation(a, b);
	}

	public static void main(String... args) {

		Calculator myApp = new Calculator();
		IntegerMath addition = (a, b) -> a + b;
		IntegerMath subtraction = (a, b) -> a - b;
		System.out.println("40 + 2 = " + myApp.operateBinary(40, 2, addition));
		System.out.println("20 - 10 = " + myApp.operateBinary(20, 10, subtraction));    
		
	}
}
\end{lstlisting}