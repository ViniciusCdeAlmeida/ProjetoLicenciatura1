\chapter{Analise estática}

A análise estática é um termo que refere-se a qualquer processo de analise de código sem executá-lo, porém teve ser passado o que exatamente está sendo procurado. Essa estrategia de análise de código possibilita mais oportunidades de verificação de estruturas e como elas se comportam do que através de uso de frameworks convencionais. Para essa checagem de código não importa se o programa está com erros ou não, o analisador somente verifica a estrutura do programa e suas variáveis.

