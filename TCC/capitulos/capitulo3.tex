\chapter{Análise estática}

A análise estática é um termo que refere-se a qualquer processo de análise de código sem executá-lo, porém deve ser determinado previamente o que ou qual comportamento de código deseja-se obter. Essa estratégia de análise possibilita uma faixa maior de oportunidades para verificação de estruturas e também ainda é possível pesquisar por comportamentos através de uso de frameworks ou tradicionais ferramentas usadas para este propósito. Para essa checagem não é relevante considerar erros ou não no programa, pois analisador verifica somente estruturas, comportamentos e suas devidas variações previamente determinadas.

