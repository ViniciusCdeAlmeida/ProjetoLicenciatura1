\chapter{Análise estática}

Análise estática é uma técnica automática no processo de verificação de software realizado por algumas ferramentas sem a necessidade de que o software tenha sido executado. Para Java exitem duas possibilidades de realizar tal análise na qual uma das técnicas realiza análise no código fonte e a outra a realiza no {\it bytecode} do programa segundo \cite{Ayewah:2008:USA:1439186.1439221}. Neste trabalho ser utilizada a pesquisa baseada no código fonte sem que tenha sido executado devido a flexibilidade e infraestrutura consolidada encontrada no eclipse AST.\\

Um fato importante é que tal análise somente obtém sucesso se forem determinados padrões ou comportamento para que sejam pesquisados no software. Neste projeto o tais comportamentos são determinados por {\it visitors} conforme explica Gamma e amigos em  \cite{Gamma:1995:DPE:186897} devido a toda infraestrutura a qual as ferramentas do eclipse fornecem facilidade para que seja realizada uma análise baseada em padrões.\\

Devido a este trabalho de verificação de software é possível detectar falhas de forma precoce nas fases de  desenvolvimento evitando que bugs e falhas sejam introduzidas e até mesmo postergados e isso é uma vantagem existe a economia de tempo com falhas simples, {\it  feedback} rápido para alertar a equipe devido as falhas ocorridas e pode-se ir além de simples casos de testes podendo aprimorar estes para que  fiquem mais rigorosos pois a partir do momento que o analisador encontrar uma falha é possível criar um teste de caso para que esta seja testada aumentando a confiabilidade do software.\\

Existe limitações nestes verificadores estáticos como em software desenvolvidos sem qualquer uso de padrões ou sem arquiteturas consolidadas, criado por equipes composta de desenvolvedores inexperientes o qual a ferramente poderá apontar erros que são falsos positivos que são erros detectados que não existem pois o analisador pesquisa por padrões e estruturas consolidadas. Tais problemas são desagradáveis porém não oferecem riscos ao desenvolvimento, podem afetar outras áreas como a de {\it refactoring} a qual poderá encontrar dificuldade em melhorar um código que não segue padrão. Vale ainda ressaltar que a penalidade de encontrar um falso positivo é a perda de tempo em fazer uma inspeção no código para comprovar se é ou não uma falha. Também há a possibilidade de falsos negativos o que cabe ao programador verificar para evitar que tais limitação do analisador não se propague durante o ciclo de desenvolvimento.\\