\section{Problema a ser Atacado}

Nos últimos anos sistemas computacionais ganharam cada vez mais espaço no mercado o que acarretou na dedicação de profissionais para manter a qualidade elevada tanto no desenvolvimento como na manutenção destes a fim de proporcionar tanto a multiplataforma quanto que qualquer equipe seja capaz de desenvolvem em qualquer local a qualquer tempo.\\

Com isso a produção de software tornou-se uma tarefa desafiadora de altíssima complexidade que pode acarretar no aumento da possibilidade de surgimento de problemas. Outro fator de grande relevância é que cada vez mais o bom desempenho do software depende da capacidade e qualificação dos profissionais que compõem a equipe de desenvolvimento. Um desses problemas é manter o desenvolvimento com partes ultrapassadas de uma linguagem o que torna um sistema obsoleto e com a chance de conter bugs e vulnerabilidades que podem comprometer a segurança de todo o sistema.\\


A atuação de equipes que desenvolvem utilizando códigos obsoletos continua sendo um grande problema no desenvolvimento de software ao longo de suas releases, mesmo com a evolução da linguagem. Códigos mais atuais tornam-se cada vez mais necessário pois evitam, corrigem falhas e vulnerabilidades além do mesmo tornar-se mais atual. Tais códigos não evoluem podem ser por falta de suporte da IDE, por falta conhecimento da equipe de desenvolvedora ou pelo simples fato de não possuir uma analisador estático que aborde estas construções lançadas nas novas versões das linguagens, especificamente java.\\


Após toda release uma linguagem demora um certo tempo de maturação para que comunidade de desenvolvedores adote novas características lançadas ou simplesmente não a utilizem, porém java possui uma filosofia de manter suporte a todos legado já desenvolvido por questão de portabilidade o que beneficia tanto IDE's quanto equipes a não ter a necessidade de se atualizarem para as ultimas versões da linguagem o que torna a construção de software com uma linguagem ultrapassada confortável porém existe a possibilidade do software possuir vulnerabilidades.\\

Um bom exemplo a ser lembrado é FORTRAN quando adicionou orientação objetos em sua release \textbf{XX} forçando a evoulução de seus compiladores os quais não forneciam mais suporte a versões anteriores conforme relata Jeffrey L. Overbey e Ralph E. Johnson em \cite{Overbey:2009:RLR:1639949.1640127}, que como consequência forçou toda comunidade desenvolvedora a se atualizar. E ainda havia a possiblidade de certos trechos de código sofrer um refectoring em tempo de compilação por um código mais atual e equivalente.\\

A processo de utilizar um analisador estático em um projeto antes de sua compilação pode vir a impactar na melhora da confiança do software pois pode detectar vulnerabilidades de maneira prematura além de reduzir o retrabalho caso estas não fossem detectadas. Tais vulnerabilidades são falhas que podem vir a ser exploradas por usuários maliciosos, estes podem desde obter acesso ao sistema, manipular dados ou até mesmo tornar todo serviço indisponível. Neste trabalho a criação de um analisador estático terá o intuito de pesquisar trechos de código ultrapassado.\\

A implementação de refectoring na grande parte das modernas IDEs mantem suporte para um simples conjunto de código onde o comportamento é intuitivo e fácil de ser analisado,  quando características avançadas de uma linguagem com o java são usados descrever precisamente o comportamento de tarefas é de extrema complexidade além da implementação do refectoring ficar complexa e de difícil entendimento segundo Max Schäfer e Oege de Moor em \cite{Schaefer:2010:SIR:1932682.1869485}. Modernas IDEs como ecplise realizam complexos refectoring através da técnica de microrefectoring que nada mais é que a divisão de um bloco de código complexo em pequenas partes para tentar encontrar códigos mais intuitivos a serem modificados.\\

O analisador estático proposto nesse trabalho tem o objeto de identificar construções ultrapassadas e porções de código congelados que são utilizadas ao logo do desenvolvimento do software verificando o histórico do lançamento das releases de software livres desenvolvidos em especialmente usando a linguagem java. Ainda caberá ao desenvolvedor tomar a decisão caso existam construções ultrapassadas nas releases se adotará o refectoring ou manterá o código congelado expondo o mesmo a usuários maliciosos.\\




