Utilizar linguagem de programação como objeto de pesquisa é uma tarefa desafiadora e complexa quer seja para minerar informações quer seja para refatorar, dada a complexidade de manipulação de uma linguagem de programação. Entretanto existe um segmento da engenharia de \textit{software} que recomenda tratar este modelo de \textit{software} como qualquer outro onde este é denominado \textit{Grammarware}. 

Partindo deste segmento, este trabalho de conclusão manipula código fonte da linguagem Java para detectar construções ultrapassadas. O principal objetivo deste trabalho foi tornar transparente a manipulação da linguagem Java para que fosse um simples \textit{input} como em qualquer outro \textit{software}. E isso mais fácil adotar esta ferramenta para checar se a linguagem em que um software qualquer está sendo desenvolvido utiliza sempre características atuais durante o desenvolvimento.

Desta forma o analisador estático que este trabalho proporcionou é capaz de pesquisar construções específicas da linguagem Java que podem ser facilmente determinadas por qualquer desenvolvedor independente da experiência na manipulação dos artefatos de uma linguagem de programação.

Para a extração dos dados este trabalho teve com principal preocupação desacoplar a extração da análise de código para que os dados minerados possam ser salvos em qualquer estrutura de dado que pode ser desde um simples arquivo~\acs{CSV} até um banco de dados.

%Atualmente encontrar blocos de código específicos tem sido de grande importância para atualizar esse trechos por um mais moderno ou mais eficiênte e assim ter os projetos utilizando sempre o que há de mais recente disponibilizado por cada \textit{feature} das linguagem no caso deste trabalho Java.

%Com isso o principal objetivo deste trabalho é criar um analisador estático com o objetivo de encontrar construções específicas na linguagem Java, contruções que podem ser código ultrapassado ou até mesmo modificações de um \textit{foreach} por uma expressão lambda. Tais contruções após encontradas farão parte de um relatório de saída para que possa ser tomada a decisão se tais contruções serão refatoradas ou não.

%Visando a maior flexibilidade possível na contrução deste analisador, a parte responsável por encontrar código fonte pré-determinado é flexível fazendo com que a qualquer momento que seja necessário possam ser criados novos visitantes sem causar impacto na estrutura do analisador. Os relatórios gerados também são flexíveis e automático podendo a qualquer momento ser modificado a geração de arquivos CSV na saída por um banco de dados caso seja de interesse do desenvolvedor.