Utilizar linguagem de programa\c{c}\~{a}o como objeto de pesquisa \'{e} uma tarefa desafiadora e complexa quer seja para minerar informa\c{c}\~{o}es quer seja para refatorar, dada a complexidade de manipula\c{c}\~{a}o de uma linguagem de programa\c{c}\~{a}o. Entretanto existe um segmento da engenharia de \textit{software} que recomenda tratar este modelo de \textit{software} como qualquer outro onde este \'{e} denominado \textit{Grammarware}. 

Partindo deste segmento, este trabalho de conclus\~{a}o manipula c\'{o}digo fonte da linguagem Java para detectar constru\c{c}\~{o}es ultrapassadas. O principal objetivo deste trabalho foi tornar transparente a manipula\c{c}\~{a}o da linguagem Java para que fosse um simples \textit{input} como em qualquer outro \textit{software}. E isso mais f\'{a}cil adotar esta ferramenta para checar se a linguagem em que um software qualquer est\'{a} sendo desenvolvido utiliza sempre caracter\'{i}sticas atuais durante o desenvolvimento.

Desta forma o analisador est\'{a}tico que este trabalho proporcionou \'{e} capaz de pesquisar constru\c{c}\~{o}es espec\'{i}ficas da linguagem Java que podem ser facilmente determinadas por qualquer desenvolvedor independente da experi\^{e}cia na manipula\c{c}\~{a}o dos artefatos de uma linguagem de programa\c{c}\~{a}o.

Para a extra\c{c}\~{a}o dos dados este trabalho teve com principal preocupa\c{c}\~{a}o desacoplar a extra\c{c}\~{a}o da an\'{a}lise de c\'{o}digo para que os dados minerados possam ser salvos em qualquer estrutura de dado que pode ser desde um simples arquivo~\acs{CSV} at\'{e} um banco de dados.

%Atualmente encontrar blocos de código específicos tem sido de grande importância para atualizar esse trechos por um mais moderno ou mais eficiênte e assim ter os projetos utilizando sempre o que há de mais recente disponibilizado por cada \textit{feature} das linguagem no caso deste trabalho Java.

%Com isso o principal objetivo deste trabalho é criar um analisador estático com o objetivo de encontrar construções específicas na linguagem Java, contruções que podem ser código ultrapassado ou até mesmo modificações de um \textit{foreach} por uma expressão lambda. Tais contruções após encontradas farão parte de um relatório de saída para que possa ser tomada a decisão se tais contruções serão refatoradas ou não.

%Visando a maior flexibilidade possível na contrução deste analisador, a parte responsável por encontrar código fonte pré-determinado é flexível fazendo com que a qualquer momento que seja necessário possam ser criados novos visitantes sem causar impacto na estrutura do analisador. Os relatórios gerados também são flexíveis e automático podendo a qualquer momento ser modificado a geração de arquivos CSV na saída por um banco de dados caso seja de interesse do desenvolvedor.

