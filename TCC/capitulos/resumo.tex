Atualmente encontrar trechos de código específicos tem sido de grande importância para refatoração desses trechos por estruturas mais modernas ou até mais eficientes e assim ter os projetos utilizando sempre o que há de mais recente disponibilizado por cada \textit{feature} da linguagem no caso deste trabalho Java.

Com isso o principal objetivo deste trabalho é criar um analisador estático flexível e estável com o objetivo de encontrar construções específicas na linguagem Java. Essas construções podem ser trechos de código obsoletos, duplicados ou até mesmo modificações de um \textit{foreach} por uma expressão lambda como exemplo. Tais estruturas após encontradas farão parte de um relatório de saída para que possa ser tomada a decisão se tais construções serão refatoradas ou não.

Visando a maior flexibilidade possível na construção deste analisador, a parte responsável por encontrar código fonte pré-determinado é flexível fazendo com que a qualquer momento que seja necessário possam ser criados novos visitantes(\textit{visitors}) sem causar impacto na estrutura do analisador. Os relatórios gerados também são flexíveis e automáticos podendo a qualquer momento ser modificado a geração de arquivos CSV na saída por um banco de dados caso seja de interesse do desenvolvedor.

