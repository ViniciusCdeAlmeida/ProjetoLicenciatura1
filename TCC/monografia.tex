%%%%%%%%%%%%%%%%%%%%%%%%%%%%%%%%%%%%%%%%
% Classe do documento
%%%%%%%%%%%%%%%%%%%%%%%%%%%%%%%%%%%%%%%%

% Nós usamos a classe "unb-cic".  Deixe apenas uma das linhas
% abaixo não-comentada, dependendo se você for do bacharelado ou
% da licenciatura.

%\documentclass[bacharelado]{unb-cic}
\documentclass[licenciatura]{unb-cic}


%%%%%%%%%%%%%%%%%%%%%%%%%%%%%%%%%%%%%%%%
% Pacotes importados
%%%%%%%%%%%%%%%%%%%%%%%%%%%%%%%%%%%%%%%%

\usepackage[brazil,american]{babel}
\usepackage[T1]{fontenc}
\usepackage{indentfirst}
\usepackage{natbib}
\usepackage{xcolor,graphicx,url}
\usepackage[utf8]{inputenc}




%%%%%%%%%%%%%%%%%%%%%%%%%%%%%%%%%%%%%%%%
% Cores dos links
%%%%%%%%%%%%%%%%%%%%%%%%%%%%%%%%%%%%%%%%

% Veja o arquivos cores.tex se quiser ver que outras cores estão
% pré-definidas.  Utilizando o comando \hypersetup abaixo nós
% evitamos aquelas caixas vermelhas feias em volta dos links.

\input{cores}
\hypersetup{
  colorlinks=true,
  linkcolor=DarkScarletRed,
  citecolor=DarkScarletRed,
  filecolor=DarkScarletRed,
  urlcolor= DarkScarletRed
}



%%%%%%%%%%%%%%%%%%%%%%%%%%%%%%%%%%%%%%%%
% Informações sobre a monografia
%%%%%%%%%%%%%%%%%%%%%%%%%%%%%%%%%%%%%%%%

\title{Título da monografia}

\orientador{\prof \dr Rodrigo Bonifácio de Almeida}{CIC/UnB}
%\coorientador[a]{\prof[a] \dr[a] Coorientadora}{MAT/UnB}
\coordenador{\prof \dr Wilson Henrique Veneziano}{CIC/UnB}
\diamesano{31}{março}{2015}

\membrobanca{\prof \dr Professor I}{CIC/UnB}
\membrobanca{\prof \dr Professor II}{CIC/UnB}

\autor{Thiago Gomes}{Cavalcanti}
\coautor{Vinícius Correa}{de Almeida}

\CDU{004.4}

\palavraschave{language design, language evolution, refactoring, microrefactoring, java}
\keywords{language design, language evolution, refactoring, microrefactoring, java}


%%%%%%%%%%%%%%%%%%%%%%%%%%%%%%%%%%%%%%%%
% Texto
%%%%%%%%%%%%%%%%%%%%%%%%%%%%%%%%%%%%%%%%

\begin{document}
  \maketitle
  \pretextual
  
  %\cite{Jeffrey_Ralph}
%\cite{Max_Oege}

  \begin{dedicatoria}
	  Dedicamos este trabalho a nossa família e ao departamento de Ciência da Computação da UnB. Que este seja apenas uma ideia inicial para que outros alunos possam ajudar a enriquecer ainda mais este projeto para que a Universidade de Brasília tenha sua própria ferramenta de análise de código e que sirva de modelo para outras Universidades.

  \end{dedicatoria}

  \begin{agradecimentos}
	Com imensa dificuldade de agradecer a tantas pessoas que de certo modo nos ajudaram nessa conquista, hora em momentos calmos hora apreensivos. Em especial a toda nossa família por dar todo suporte necessário para que pudessemos concluir essa etapa em nossas vidas, também aluna Daniela Angellos pelo seu desdobramento e conhecimento para nos ajudar a criar essa ferramenta.\\
Em especial ao professor dr. Rodrigo Bonifácio que nos inseriu nesse imenso mundo da Engenharia de Software, hora apresentando um problemática hora ajundando a resolver barreiras as quais não conseguimos sozinhos.\\
E ainda a UnB por todo seu corpo docente que sem este essa jornada não seria concluida com excelência, em especial aoprofessor dr. Edson Alves da Costa Júnior por se deslocar da UnB-Gama para nos ajudar.\\

  \end{agradecimentos}

  \begin{resumo}
	Atualmente encontrar trechos de código específicos tem sido de grande importância para Utilizar linguagem de programa\c{c}\~{a}o como objeto de pesquisa \'{e} uma tarefa desafiadora e complexa quer seja para minerar informa\c{c}\~{o}es quer seja para refatorar, dada a complexidade de manipula\c{c}\~{a}o de uma linguagem de programa\c{c}\~{a}o. Entretanto existe um segmento da engenharia de \textit{software} que recomenda tratar este modelo de \textit{software} como qualquer outro onde este \'{e} denominado \textit{Grammarware}. 

Partindo deste segmento, este trabalho de conclus\~{a}o manipula c\'{o}digo fonte da linguagem Java para detectar constru\c{c}\~{o}es ultrapassadas. O principal objetivo deste trabalho foi tornar transparente a manipula\c{c}\~{a}o da linguagem Java para que fosse um simples \textit{input} como em qualquer outro \textit{software}. E isso mais f\'{a}cil adotar esta ferramenta para checar se a linguagem em que um software qualquer est\'{a} sendo desenvolvido utiliza sempre caracter\'{i}sticas atuais durante o desenvolvimento.

Desta forma o analisador est\'{a}tico que este trabalho proporcionou \'{e} capaz de pesquisar constru\c{c}\~{o}es espec\'{i}ficas da linguagem Java que podem ser facilmente determinadas por qualquer desenvolvedor independente da experi\^{e}cia na manipula\c{c}\~{a}o dos artefatos de uma linguagem de programa\c{c}\~{a}o.

Para a extra\c{c}\~{a}o dos dados este trabalho teve com principal preocupa\c{c}\~{a}o desacoplar a extra\c{c}\~{a}o da an\'{a}lise de c\'{o}digo para que os dados minerados possam ser salvos em qualquer estrutura de dado que pode ser desde um simples arquivo~\acs{CSV} at\'{e} um banco de dados.


  \end{resumo}


  \selectlanguage{american}
  \begin{abstract}
  	Use programming language like a research subject is a challenge and complex task either to mine information or to refactor, given the complexity of handling a programming language. However there is a segment of software engineering which recommends treating this model software like any other where this is called Grammarware.

From this segment, this final project handles source code of the Java language to detect outdated buildings. The main objective was to provide transparency in the handling of the Java language to make it a simple input as any other software. And easier to adopt this tool to check whether the language in which any software is being developed always uses current characteristics during development.

Thus the static analyzer that this work provided it is able to search for specific constructs of the Java language that can be easily determined by any independent developer of experience in handling the artifacts of a programming language.

For the extraction of data this work was with main concern to separate the extraction of code analysis and the mined data can be saved to any data structure that can be anything from a single file~\acs{CSV} accessible to a database.


%Search to specific code has been very important from update to a more actual or efficient and with the projetc has every the least release of a language at this work Java.

%Therefore the main goal of this project is develop a static analysis with objective to find specifics constructions of Java language, where this constructions can be older code or a update a block to another better such as foreach for a lambda expression. After find this code the place in source code is saved to write a output file for future evaluation and decide if this will be updated or not.

%With focus in a flexibility the project the party responsible for visitors that find source code previously determined is the highest flexible that make easy in any time the developer create their own visitor and insert in the system without impacts in architecture. The output reports are flexible and automatics that provide in any time a possibility of chance the actuals ~\acs{CSV} files to another form such as database.
  \end{abstract}
  
  
  \selectlanguage{brazil}

  \tableofcontents
  \listoffigures
  \listoftables

  \textual
  \include{capitulos/capitulo1_Introducao}
\include{capitulos/capitulo1_historia_da_linguagem}
\section {Aspectos evolutivos da liguagem Java}
		\subsection {Java 2}
			A primeira versão do Java Security, disponível no JDK 1.1 \cite{JDK1.1}, contém um subconjunto dessa funcionalidade, incluindo APIs para:
		  \begin{itemize}
			  \item Assinaturas Digitais: Algoritmos de assinatura digital, como DSA ou MD5 com RSA. A funcionalidade inclui a geração de chaves público/privado , bem como assinatura e verificação de dados digitais.
			  \item Gerenciamento de Chaves: Um conjunto de abstrações para o gerenciamento de ''diretores'' (entidades como usuários individuais ou grupos), suas chaves, e os seus certificados. Ele permite que aplicativos para projetar seu próprio sistema de gerenciamento de chaves, e para interoperar com outros sistemas em alto nível.
			  \item Lista de controle de acesso: Um conjunto de abstrações para o gerenciamento de ''diretores'' e suas permissões de acesso.
			  \item A obtenção de um objeto de assinatura: 
			  
\begin{lstlisting}
import java.security.Signature;
import java.security.NoSuchAlgorithmException;
	
public class SignFile {
	Signature signature;
		
	private void init(String algorithm) throws NoSuchAlgorithmException{
		signature = Signature.getSignature(algorithm);
    }
}
\end{lstlisting}
			  
			  \item Em versões anteriores, Java suportava apenas {\it top-level} classes, que devem ser membros de pacotes. Na versão 1.1, o programador Java pode agora definir classes internas como membros de outras classes \cite{bracha1998gj}, localmente dentro de um bloco de instruções, ou (anonimamente) dentro de uma expressão.
		  
\begin{lstlisting}
public class FixedStack {
	...
	 public java.util.Enumeration elements() {
	     return new FixedStack$Enumerator(this);
	 }
}
		
class FixedStack$Enumerator implements java.util.Enumeration {
	private FixedStack this$0;
	
	FixedStack$Enumerator(FixedStack this$0) {
		this.this$0 = this$0;
		this.count = this$0.top;
	 }
			
	int count;
	public boolean hasMoreElements() {
		return count > 0;
	}
		
	public Object nextElement() {
		if (count == 0)
			throw new NoSuchElementException("FixedStack");
		
		return this$0.array[--count];
	}
}
\end{lstlisting}
			
			\clearpage
			\item Para escrever um objeto remoto (RMI), você escrever uma classe que implementa uma ou mais interfaces remotas. 
			
\begin{lstlisting}
package examples.hello;
public interface Hello extends java.rmi.Remote {
	String sayHello() throws java.rmi.RemoteException;
}
\end{lstlisting}
	 
\item HelloImpl.java
\begin{lstlisting}
package examples.hello;

import java.rmi.;
import java.rmi.server.UnicastRemoteObject;

public class HelloImpl extends UnicastRemoteObject implements Hello{
	private String name;
	
	public HelloImpl(String s) throws RemoteException {
		super();
		name = s;
	}
	
	public String sayHello() throws RemoteException {
		return  "Hello World!";
	}
	
	public static void main(String args[]){
	
		System.setSecurityManager(new RMISecurityManager());
	
		try {
			HelloImpl obj = new HelloImpl("HelloServer");
			Naming.rebind("//myhost/HelloServer", obj);
			System.out.println("HelloServer bound in registry");
		} catch (Exception e) {
			System.out.println("HelloImpl err: " + e.getMessage());
			e.printStackTrace();
		}
	}
}
\end{lstlisting}
		\end{itemize}


	\clearpage
	\subsection {Java 4}
	  \begin{itemize}
		  \item {\it Assertion Facility} \cite{JSE8_Enhancements}. As {\it assertions} são expressões booleanas que o programador acredita ser verdade sobre o estado de um programa de computador. Por exemplo, depois de ordenar uma lista o programador pode afirmar que a lista está em ordem crescente. Avaliando as afirmações em tempo de execução para confirmar a sua validade é uma das ferramentas mais poderosas para melhorar a qualidade do código, uma vez que rapidamente se descobre equívocos do programador sobre o comportamento de um programa.
	  \end{itemize}
	
	\subsection {Java 5}
	  \begin{itemize}
		  \item {\it Generics}\cite{JSE8_Enhancements, OracleGenerics,Parnin:2011:JGA:1985441.1985446}. Este novo recurso para o sistema de tipo permite que um tipo ou método operar em objetos de vários tipos, proporcionando em tempo de compilação tipo de segurança. Acrescenta em tempo de compilação um tipo de segurança para as {\it collections} e elimina o trabalho penoso de {\it casting}. Um exemplo do uso de {\it colletions} e {\it generics} respectivamente:
\begin{lstlisting}
static void expurgate(Collection c) {
	for (Iterator i = c.iterator(); i.hasNext(); )
		if (((String) i.next()).length() == 4)
			i.remove();
	}
	
static void expurgate(Collection<String> c) {
	for (Iterator<String> i = c.iterator(); i.hasNext(); )
		if (i.next().length() == 4)
			i.remove();
}
\end{lstlisting}
		  
		\item {\it For-Each Loop}. Esta nova estrutura de linguagem elimina o trabalho e erro de propensão de iteradores e variáveis de índice quando a iteração ocorre sobre coleções e arrays. Como a construção evoluiu com o advento dessa nova estrutura:
	
\begin{lstlisting}
void cancelAll(Collection<TimerTask> c) {
	for (Iterator<TimerTask> i = c.iterator(); i.hasNext(); )
	   i.next().cancel();
}
	
void cancelAll(Collection<TimerTask> c) {
	for (TimerTask t : c)
		t.cancel();
}
\end{lstlisting}
	  
	  \clearpage
	  \item {\it Varargs}. Esta nova estrutura tende a eliminar a necessidade de passagem manual de listas de argumentos em um array ao invocar métodos que aceitam de um comprimento variável de uma lista de argumentos. Nas versões anteriores, um método levava um número arbitrário de valores necessários a  criar uma matriz e colocar os valores para a matriz antes de chamar o método.


\begin{lstlisting}
public class Test {
	public static void main(String[] args) {
	  int passed = 0;
	  int failed = 0;
	  for (String className : args) {
	      try {
	          Class c = Class.forName(className);
	          c.getMethod("test").invoke(c.newInstance());
	          passed++;
	      } catch (Exception ex) {
	          System.out.printf("%s failed: %s%n", className, ex);
	          failed++;
	      }
	  }
	  System.out.printf("passed=%d; failed=%d%n", passed, failed);
	}
}
\end{lstlisting}
 
	  \item {\it Autoboxing/Unboxing}. Esta nova estrutura elimina o trabalho de conversão manual entre tipos primitivos (como {\it int}) e os tipos de classes {\it wrapper}
  \end{itemize}
  
  
  
	\subsection {Java 6}
		Não ocorram mudanças ou introdução de novas estruturas na linguagem Java \cite{JSE8_Enhancements}.
	
	\subsection {Java 7}
		\begin{itemize}
		  \item {\it Multi Catch} e lançamento de exceções com melhora na verificação de tipos. Um único bloco {\it catch} poderá lidar com mais de um tipo de exceção. Além disso, o compilador executa a análise mais precisa das exceções. Isso permite que o programador especifique tipos de exceção mais específicos na cláusula de uma declaração método. Um exemplo de como era as estruturas que usavam {\it cacths} e com a introdução de {\it multi catch} com o Java 7 \cite{JSE7}, respectivamente.
  

\begin{lstlisting}
catch (IOException ex) {
	logger.log(ex);
	throw ex;
}catch (SQLException ex) {
	logger.log(ex);
	throw ex;
}
\end{lstlisting}
\clearpage
\begin{lstlisting}
catch (IOException|SQLException ex) {
	logger.log(ex);
	throw ex;
}
\end{lstlisting}


 \item O {\it try-with-resouces}. A declaração {\it try-with-resouces} é uma instrução {\it try} que declara um ou mais recursos. Um recurso é um objeto que deve ser fechada após o programa terminar com ele. Essa declaração garante que cada recurso é fechada no final da declaração.
	 
\begin{lstlisting}

public static void writeToFileZipFileContents(
			String zipFileName, String outputFileName) throws java.io.IOException {
	
	java.nio.charset.Charset charset = java.nio.charset.StandardCharsets.US_ASCII;
	java.nio.file.Path outputFilePath = java.nio.file.Paths.get(outputFileName);
	
	try(
		 java.util.zip.ZipFile zf = new java.util.zip.ZipFile(zipFileName);
		 java.io.BufferedWriter writer = java.nio.file.Files.newBufferedWriter(outputFilePath, charset)
	){
	
		for (java.util.Enumeration entries = zf.entries(); entries.hasMoreElements();) {
			 String newLine = System.getProperty("line.separator");
			 String zipEntryName = ((java.util.zip.ZipEntry)entries.nextElement()).getName() + newLine;
			 writer.write(zipEntryName, 0, zipEntryName.length());
		 }
	}
}

\end{lstlisting}
		  
		  \clearpage
		  \item Inferência de tipos para criação de instâncias em {\it generics}\cite{OracleGenerics}\cite{Bracha:1998:MFS:286942.286957}\cite{Parnin:2011:JGA:1985441.1985446}. Com o Java 7 pode-se substituir os argumentos de tipo necessários para invocar o construtor de uma classe genérica com um conjunto vazio de parâmetros de tipo (<>), desde que o compilador infira os argumentos de tipo a partir do contexto. Este par de colchetes angulares é informalmente chamado de diamante.
  
  

\begin{lstlisting}
Map<String, List<String>> myMap = new HashMap<String, List<String>>();
Map<String, List<String>> myMap = new HashMap<>();
	
List<String> list = new ArrayList<>();
list.add("A");

list.addAll(new ArrayList<>());
	
class MyClass<X> {
	<T> MyClass(T t) {
	...
	}
}
\end{lstlisting}
	 
	  \end{itemize}
	  
	  
	\subsection{Java 8}
	  \begin{itemize}
		  \item Melhoria na inferência de tipos. O compilador Java aproveita digitação para inferir os parâmetros de tipo de uma invocação de método genérica. O tipo de destino de uma expressão é o tipo de dados que o compilador Java espera, dependendo de onde a expressão aparece. Por exemplo, pode-se usar o tipo de destino de uma instrução de atribuição para o tipo de inferência em Java 7. No entanto, em Java 8, pode-se usar o tipo de destino para a inferência de tipos em mais contextos. O exemplo mais proeminente está usando tipos de destino de um método de invocação para inferir os tipos de dados dos seus argumentos.

\begin{lstlisting}
	List<String> stringList = new ArrayList<>();
	stringList.add("A");
	stringList.addAll(Arrays.asList());
\end{lstlisting}
		  
		  
		  \item Expressões lambda. Permitem encapsular uma única unidade de comportamento e passá-lo para outro código. Pode-se usar uma expressãos lambda, se quiser uma determinada ação executada em cada elemento de uma {\it collection}, quando o processo for concluído, ou quando um processo encontra um erro. \cite{JSE7} \\
	  \end{itemize}

\clearpage
\begin{lstlisting}
public class Calculator {
	
	interface IntegerMath {
		int operation(int a, int b);   
	}

	public int operateBinary(int a, int b, IntegerMath op) {
		return op.operation(a, b);
	}

	public static void main(String... args) {

		Calculator myApp = new Calculator();
		IntegerMath addition = (a, b) -> a + b;
		IntegerMath subtraction = (a, b) -> a - b;
		System.out.println("40 + 2 = " + myApp.operateBinary(40, 2, addition));
		System.out.println("20 - 10 = " + myApp.operateBinary(20, 10, subtraction));    
		
	}
}
\end{lstlisting}
\chapter{Problema a ser Atacado}
\section{Problematização}

Nos últimos anos sistemas computacionais ganharam cada vez mais espaço no mercado o que acarretou na dedicação de profissionais para manter a qualidade elevada tanto no desenvolvimento como na manutenção destes a fim de proporcionar tanto a multiplataforma quanto que qualquer equipe seja capaz de desenvolvem em qualquer local a qualquer tempo.\\

Com isso a produção de software tornou-se uma tarefa desafiadora de altíssima complexidade que pode acarretar no aumento da possibilidade de surgimento de problemas. Outro fator de grande relevância é que cada vez mais o bom desempenho do software depende da capacidade e qualificação dos profissionais que compoẽm a equipe de desenvolvimento. Um desses problemas é manter o desenvolvimento com partes ultrapassadas de uma linguagem o que torna um sistema obsoleto e com a chance de conter bugs e vulnerabilidades que podem comprometer a segurança de todo o sistema.\\


A atuação de equipes que desenvolvem utilizando códigos obsoletos continua sendo um grande problema no desenvolvimento de software ao longo de suas releases, mesmo com a evolução da linguagem. Códigos mais atuais tornam-se cada vez mais necessário pois evitam, corrigem falhas e vulnerabilidades além do mesmo tornar-se mais atual. Tais códigos não evoluem podem ser por falta de suporte da IDE, por falta conhecimento da equipe de desenvolvedora ou pelo simples fato de não possuir uma analisador estático que aborde estas construções lançadas nas novas versões das linguagens, especificamente java.\\


Após toda release uma linguagem demora um certo tempo de maturação para que comunidade de desenvolvedores adote novas características lançadas ou simplesmente não a utilizem, porém java possui uma filosofia de manter suporte a todos legado já desenvolvido por questão de portabilidade o que beneficia tanto IDE's quanto equipes a não ter a necessidade de se atualizarem para as ultimas versões da linguagem o que torna a construção de software com uma linguagem ultrapassada confortável porém existe a possibilidade do software possuir vulnerabilidades.\\

Um bom exemplo a ser lembrado é FORTRAN quando adicionou orientação objetos em sua release \textbf{XX} forçando a evoulução de seus compiladores os quais não forneciam mais suporte a versões anteriores conforme relata Jeffrey L. Overbey e Ralph E. Johnson em \cite{Overbey:2009:RLR:1639949.1640127}, que como consequência forçou toda comunidade desenvolvedora a se atualizar. E ainda havia a possiblidade de certos trechos de código sofrer um refectoring em tempo de compilação por um código mais atual e equivalente.\\

A processo de utilizar um analisador estático em um projeto antes de sua compilação pode vir a impactar na melhora da confiança do software pois pode detectar vulnerabilidades de maneira prematura além de reduzir o retrabalho caso estas não fossem detectadas. Tais vulnerabilidades são falhas que podem vir a ser exploradas por usuários maliciosos, estes podem desde obter acesso ao sistema, manipular dados ou até mesmo tornar todo serviço indisponível. Neste trabalho a criação de um analisador estático terá o intuito de pesquisar trechos de código ultrapassado.\\

A implementação de refectoring na grande parte das modernas IDEs mantem suporte para um simples conjunto de código onde o comportamento é intuitivo e fácil de ser analisado,  quando características avançadas de uma linguagem com o java são usados descrever precisamente o comportamento de tarefas é de extrema complexidade além da implementação do refectoring ficar complexa e de difícil entendimento segundo Max Schäfer e Oege de Moor em \cite{Schaefer:2010:SIR:1932682.1869485}. Modernas IDEs como ecplise realizam complexos refectoring através da técnica de microrefectoring que nada mais é que a divisão de um bloco de código complexo em pequenas partes para tentar encontrar códigos mais intuitivos a serem modificados.\\

O analisador estático proposto nesse trabalho tem o objeto de identificar construções ultrapassadas e porções de código congelados que são utilizadas ao logo do desenvolvimento do software verificando o histórico do lançamento das releases de software livres desenvolvidos em especialmente usando a linguagem java. Ainda caberá ao desenvolvedor tomar a decisão caso existam construções ultrapassadas nas releases se adotará o refectoring ou manterá o código congelado expondo o mesmo a usuários maliciosos.\\







  \chapter{Java}
\section {Historia da linguagem}

No começo da decada de 90 um pequeno grupo de engenheros da Oracle chamados de ''Green Team'' acreditava que a próxima onde de na area da computação seria a união de equipamentos eletroeletrônicos com os computadores. O ''Green Team'' liderado por James Gosling, demonstraram que a linguagem de programaçao Java, que foi desenvolvida pela equipe e originalmente era chamado de Oak, foi desenvolvida para dispositivos de entretenimento como aparelhos de tv a cabo, porem não foi bem aceita no meio. Em 1995 com a massificação da Internet foi quando a linguagem Java teve sua primeira grande aplicação o navegador Netscape.\\

Java é uma linguagem de programação de propósito geral orientada a objetos, concebida especificadademente para ter poucas dependencias de implementação que isso acarreta que uma vez que a aplicação fora desenvolvida ela poderá ser executada em qualquer lugar.\\

Na sua primeira versão chamada de Java 1 (JDK* 1.0.2) onde introduziram oito pacotes básicos do java como: java.lang, java.io, java.util, java.net, java.awt, java.awt.image, java.awt.peer e java.applet. Foi usado para o desenvolvimento de ferramentas populares na epoca como o Netscape 3.0 e o Internet Explorer 3.0. \\

Sua segunda versão foi o JDK* 1.1 onde trouxe ganhos em funcionalidades, desempenho e qualidade. Novas aplicações tambem surgiram como : JavaBeans, aprimoramento do AWT*, novas funcionalidades como o JDBC*, acesso remoto ao objeto (RMI*) e suporte ao padrão Unicode 2.0.\\

Na terceira versão Java 2 (JDK* 1.2) ofereceu melhorias significativas no desempenho, um novo modelo de segurança, flexível e um conjunto completo de aplicações de programação interfaces (APIs). Os novos recursos da plataforma Java 2 incluiram: 
\begin{itemize}
  \item O modelo de "sandbox"  foi ampliado para dar aos desenvolvedores, usuários e administradores de sistema a opção de especificar e gerenciar um conjunto de políticas de segurança flexíveis que governam as ações de uma aplicação ou applet que pode ou não ser executada.
  \item Suporte nativo a thread para o ambiente operacional Solaris. Compressão de memória para classes carregadas. Alocação de memória com mais desempenho e melhor para a coleta de lixo. Arquitetura de máquina virtual conectável para outras máquinas virtuais, incluindo a Java HotSpot VMNew. Just in Time (JIT*). Java Native Interface (JNI*) de conversão.
  \item O conjunto de componentes de projeto, GUI (Swing). API Java 2D que fornece novos recursos gráficos 2D e AWT*, bem como suporte para impressão. O Java {\it look and fell} de interface. Uma nova API de acessibilidade.
  \item Framework de entrada de caracteres (suporte a japonês, chinês e coreano). Complexo de saída usando a API* do Java 2D para fornecer um {\it display} bi-direcional, de alta qualidade de japonês, árabe, hebraico e outras línguas de caracteres.
  \item Java Plug-in para navegadores da Web, incluída na plataforma Java 2, fornecendo um tempo de execução totalmente compatível com a máquina virtual Java amplamente implantadas em navegadores.
  \item Invocação das operações ou serviços de rede remoto. Totalmente compatível com Java ORB e incluído no tempo de execução.
  \item JDBC que fornece um acesso mais fácil aos dados para consultas mais flexíveis. Melhor desempenho e estabilidade são promovidos por cursores de rolagem e suporte para SQL3 de tipos.\\
\end{itemize}

Em 8 de maio de 2000 foi anunciado o Java 2 versão 1.3 que trouxe ganho de desempenho em relação a primeira versão da JS2E* de cerca de 40\%  no tempo de {\it  start-up} e de 20\%. Tambem trouxe novas funcionaliadades como: 

\begin{itemize}
  \item O Java HotSpot VM* de cliente e suas bibliotecas atentando ao desempenho ao fazer o J2SE* versão 1.3 a {\it realease} o mais rápido até à data.
  \item Novos recursos, como o {\it caching applet} e instalação do pacote opcional Java através da tecnologia Java {\it  Plug-in} para aumentar a velocidade e a flexibilidade com que os {\it applets} e aplicativos baseados na tecnologia Java pode ser implantado. Java {\it  Plug-in} tecnologia é um componente do ambiente de execução Java 2, Standard Edition v 1.3 que permite Java {\it applets} e aplicativos para a execução.
  \item O novo suporte para RSA* assinatura eletrônica, gerenciamento de confiança dinâmico, certificados X.509, e verificação de arquivos o que significa o aumento das possibilidades que os desenvolvedores tem para proteger dados eletrônicos.
  \item Uma série de novos recursos e ferramentas de desenvolvimento da tecnologia J2SE* versão 1.3 que permite o desenvolvimento mais fácil e rápido de aplicações baseadas na tecnologia {\it web} ou Java {\it  standalone} de alto desempenho.
  \item A adição de RMI / IIOP* e o Jndi (JNDI*) para a versão 1.3, melhora na interoperabilidade J2SE*. RMI / IIOP* melhora a conectividade com sistemas de {\it  back-end} que suportam CORBA*. JNDI fornece acesso aos diretórios que suportam o populares LDAP* Lightweight Directory Access Protocol, entre outros.\\
\end{itemize}

No ano de 2000 no dia 6 de Fevereiro, foi lançado a J2SE* versão 1.4. Com a versão 1.4, as empresas puderam usar a tecnologia Java para desenvolver aplicativos de negócios mais exigentes e com menos esforço e em menos tempo. As novas funcionalidades como a nova I/O* e suporte a 64 bits. A J2SE* se tornou plataforma ideal para a mineração em grande escala de dados, inteligência de negócios, engenharia e científicos. A versão 1.4 forneceu suporte aprimorado para tecnologias padrões da indústria, tais como SSL*, LDAP* e CORBA* a fim de garantir a operacionalidade em plataformas heterogêneas, sistemas e ambientes. Com o apoio embutido para XML*, a autenticação avançada, e um conjunto completo de serviços de segurança, está versão forneceu base para padrões de aplicações Web e serviços interoperáveis. O J2SE* avançou o desenvolvimento de aplicativos de cliente com novos controles de GUI, acelerou Java 2D, a performance gráfica, internacionalização e localização expandida de apoio, novas opções de implantação e suporte expandido para o até então Windows XP.\\

Com a chegada da JSE2* versão 1.5 (Java 5.0) em 6 de ferevereiro de 2002, impulsionou benefícios extensivos para desenvolvedores, incluindo a facilidade de uso, desempenho global e escalabilidade, monitoramento do sistema e gestão e desenvolvimento. O Java 5 foi derivado do trabalho de 15 componentes Java Specification Requests (JSRs) englobando recursos avançados para a linguagem e plataforma. Os líderes da indústria na época que participam no grupo de peritos J2SE 5.0 incluiram: Apache Software Foundation, Apple Computer, BEA Systems, Borland Software Corporation, Cisco Systems, Fujitsu Limited, HP, IBM, Macromedia, Nokia Corporation, Oracle, SAP AG, SAS Institute, SavaJe Technologies e Sun Microsystems.

Novas funcionalidades foram implementadas como:

\begin{itemize}
  \item Facilidade de desenvolvimento: os programadores da linguagem Java pode ser mais eficiente e produtivos com os recursos de linguagem Java 5 que permitiram a codificação mais segura. Nesta versão surgiu {\it Generics}, tipos enumerados, metadados e autoboxing de tipos primitivos permitindo assim uma fácil e rápida codificação.
  \item Monitoramento e gestão: Um foco chave para a nova versão da plataforma, a aplicativos baseados na tecnologia Java {\it Virtual Machine} que passou a ser monitorado e gerenciado com o {\it built-in} de suporte para Java {\it Management Extensions}. Isso ajudou a garantir que seus funcionários, sistemas de parceiros do cliente permanecessem em funcionamento por mais tempo. Suporte para sistemas de gestão empresarial baseados em SNMP* também é viável.
  \item Um olhar novo aplicativo, mais moderna, baseada na tecnologia Java padrão e proporciona uma sensação GUI para aplicativos baseados na tecnologia Java. A J2SE* 5.0 teve suporte completo a internacionalização e também possuindo suporte para aceleração de hardware por meio da API OpenGL* e tambem para o sistema operacional Solaris e sistemas operacionais da distribuição Linux.
  \item Maior desempenho e escalabilidade: A nova versão incluiu melhorias de desempenho, tais como menor tempo de inicialização, um menor consumo de memória e JVM* auto ajustável para gerar maior desempenho geral do aplicativo e desenvolvimento em J2SE 5.0 em relação às versões anteriores.\\
\end{itemize}

Java 1.6 (Java 6) foi divulgado em 11 de dezembro de 2006. Tornou o desenvolvimento mais fácil, mais rápido e mais eficiente em termos de custos e ofereceu funcionalidades para serviços web, suporte linguagem dinâmica, diagnósticos e aplicações desktop. Com a chegada dessa nova versão do Java houve combinação com o NetBeans IDE 5.5 fornecendo aos desenvolvedores uma estrutura confiável, de codigo aberto e compatível, de alta performance para entregar aplicativos baseados na tecnologia Java mais rápido e mais fácil do que nunca. O NetBeans IDE* fornece uma fonte aberta e de alto desempenho, modular, extensível, multi-plataforma Java IDE* para acelerar o desenvolvimento de aplicações baseadas em software e serviços {\it web}.
Novas funcionalidades foram implementadas como:

\begin{itemize}
  \item O Java 1.6 ajudou a acelerar a inovação para o desenvolvedor, aplicativos de colaboração {\it online} e baseadas na {\it web}, incluindo um novo quadro de desenvolvedores APIs para permitir a mistura da tecnologia Java com linguagens de tipagem dinâmica, tais como PHP, Python, Ruby e tecnologia JavaScript. A Sun também criou uma coleção de mecanismos de script e pré-configurado o motor JavaScript Rhino na plataforma Java. Além disso, o software inclui uma pilha completa de clientes de serviços web e suporta as mais recentes especificações de serviços {\it web}, como JAX-WS 2.0*, JAXB 2.0*, STAX* e JAXP.*
  \item A plataforma Java 1.6 forneceu ferramentas expandidas para o diagnóstico, gestão e monitoramento de aplicações e também inclui suporte para o novo NetBeans Profiler 5.5 para Solaris DTrace e, uma estrutura de rastreamento dinâmico abrangente que está incluído no sistema operacional Solaris 10. Além disso, o software Java SE 6 aumenta ainda mais a facilidade de desenvolvimento com atualizações de interface ferramenta para o Java Virtual Machine (JVM) e o Java Platform Debugger Architecture (ACDP)*.
\end{itemize}

Java 7 foi lançado no dia 28 de julho de 2011. Essa versão foi resultado do desenvolvimento de toda a indústria envolvendo uma revisão de codigo aberto e extensa colaboração entre os engenheiros da {\it Oracle} e membros do ecossistema Java em todo o mundo através da comunidade {\it OpenJDK} e do {\it Java Community Process} (JCP)*. Compatibilidade com versões anteriores de Java 7 com versões anteriores da plataforma a fim de preservar os conjuntos de habilidades dos desenvolvedores de software Java e proteger os investimentos em tecnologia Java.

Com essa versão novas funcionalidades foram adicionadas:

\begin{itemize}
  \item As alterações de linguagem ajudaram a aumentar a produtividade do desenvolvedor e simplificar tarefas comuns de programação, reduzindo a quantidade de código necessário, esclarecendo sintaxe e tornar o código com mais legibilidade.
  \item Melhor suporte para linguagens dinâmicas incluindo: Ruby, Python e JavaScript, resultando em aumentos substanciais de desempenho no JVM*.
  \item Uma nova API* {\it multicore-ready} que permite aos desenvolvedores para se decompor mais facilmente problemas em tarefas que podem ser executadas em paralelo em números arbitrários de núcleos de processador.
  \item Uma interface de I/O* abrangente para trabalhar com sistemas de arquivos que podem acessar uma ampla gama de atributos de arquivos e oferecem mais informações quando ocorrem erros.
  \item Novos recursos de rede e de segurança. Suporte expandido para a internacionalização, incluindo suporte a Unicode 6.0. Versões atualizadas das bibliotecas padrão.\\
  \item
  \item
  \item
  \item
\end{itemize}



\\

\section {Aspectos evolutivos da liguagem Java}
\subsection {Java 1}
\subsection {Java 2}
\subsection {Java 3}
\subsection {Java 4}
\subsection {Java 5}
\subsection {Java 6}
\subsection {Java 7}
\subsection {Java 8}
 



								

  \postextual
  \bibliographystyle{plain}
  \bibliography{referencias/referencias}

\end{document}
